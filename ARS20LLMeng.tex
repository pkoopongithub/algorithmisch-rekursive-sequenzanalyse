% Options for packages loaded elsewhere
\PassOptionsToPackage{unicode}{hyperref}
\PassOptionsToPackage{hyphens}{url}
\documentclass[
]{article}
\usepackage{xcolor}
\usepackage{amsmath,amssymb}
\setcounter{secnumdepth}{-\maxdimen} % remove section numbering
\usepackage{iftex}
\ifPDFTeX
  \usepackage[T1]{fontenc}
  \usepackage[utf8]{inputenc}
  \usepackage{textcomp} % provide euro and other symbols
\else % if luatex or xetex
  \usepackage{unicode-math} % this also loads fontspec
  \defaultfontfeatures{Scale=MatchLowercase}
  \defaultfontfeatures[\rmfamily]{Ligatures=TeX,Scale=1}
\fi
\usepackage{lmodern}
\ifPDFTeX\else
  % xetex/luatex font selection
\fi
% Use upquote if available, for straight quotes in verbatim environments
\IfFileExists{upquote.sty}{\usepackage{upquote}}{}
\IfFileExists{microtype.sty}{% use microtype if available
  \usepackage[]{microtype}
  \UseMicrotypeSet[protrusion]{basicmath} % disable protrusion for tt fonts
}{}
\makeatletter
\@ifundefined{KOMAClassName}{% if non-KOMA class
  \IfFileExists{parskip.sty}{%
    \usepackage{parskip}
  }{% else
    \setlength{\parindent}{0pt}
    \setlength{\parskip}{6pt plus 2pt minus 1pt}}
}{% if KOMA class
  \KOMAoptions{parskip=half}}
\makeatother
\setlength{\emergencystretch}{3em} % prevent overfull lines
\providecommand{\tightlist}{%
  \setlength{\itemsep}{0pt}\setlength{\parskip}{0pt}}
\usepackage{bookmark}
\IfFileExists{xurl.sty}{\usepackage{xurl}}{} % add URL line breaks if available
\urlstyle{same}
\hypersetup{
  hidelinks,
  pdfcreator={LaTeX via pandoc}}

\author{}
\date{}

\begin{document}

\subsubsection{\texorpdfstring{\textbf{Does the use of LLM in
qualitative social research contribute to
understanding?}}{Does the use of LLM in qualitative social research contribute to understanding?}}\label{does-the-use-of-llm-in-qualitative-social-research-contribute-to-understanding}

The statement that large-scale language models (LLMs) imitate but do not
explain dialogues is central to this question. LLMs are capable of
replicating the contingency and opacity of dialogues because they are
based on the statistical analysis of vast amounts of text and calculate
probabilities for the sequence of words and sentences. They can generate
convincingly human-sounding dialogues and even identify patterns in
qualitative data.

When qualitative social researchers outsource their work to LLMs, this
can significantly increase efficiency in data processing and pattern
recognition. LLMs can quickly search large volumes of transcripts,
cluster themes, or suggest initial category systems.

\textbf{However, this does not directly contribute to the
deeper\emph{Understand}in the sense of qualitative social research, for
the following reasons:}

\begin{itemize}
\item
  \textbf{Imitation vs. Explanation:}LLMs are essentially imitation
  machines. They reproduce existing patterns
  without\emph{Understanding}the underlying social meanings,
  motivations, contexts, or intentional actions of the actors. The "why"
  or "how" of social phenomena, which qualitative research aims to
  understand, remains inaccessible to LLM.
\item
  \textbf{Opacity of the LLM:}The functioning of LLMs is itself highly
  opaque ("black box"). While they produce results, the path to them is
  not transparent or comprehensible to the researcher in terms of human
  interpretation.
\item
  \textbf{Lack of critical reflection:}Qualitative social research
  requires the researcher to critically reflect on their own
  assumptions, the research process, and the social implications of the
  results. LLMs cannot provide this level of reflection.
\item
  \textbf{Contingency of LLM results:}Although LLMs can mimic
  contingency in dialogues, their own results are contingent with
  respect to the training data and algorithms, which limits the
  generalizability and theoretical foundation of their "insights."
\end{itemize}

The use of LLM can be a\textbf{valuable tool}to prepare, structure, and
support qualitative analysis by automating certain tasks and offering
new perspectives on the data. However, the actual understanding remains
the domain of the human researcher, who must interpret, contextualize,
and theoretically classify the patterns generated by LLM. Without this
human interpretation, the results of LLM remain merely a complicated
form of pattern recognition.

\subsubsection{\texorpdfstring{\textbf{Comparison with Algorithmic
Recursive Sequence Analysis (ARS) and whether its results are more
explanatory:}}{Comparison with Algorithmic Recursive Sequence Analysis (ARS) and whether its results are more explanatory:}}\label{comparison-with-algorithmic-recursive-sequence-analysis-ars-and-whether-its-results-are-more-explanatory}

Algorithmic Recursive Sequence Analysis 2.0 (ARS 2.0), as described in
the uploaded documents, is fundamentally different from LLM and can be
considered as\textbf{rather explanatory model}be considered.

\textbf{Comparison points:}

\begin{itemize}
\item
  \textbf{Focus on grammars:}ARS 2.0 aims to provide a\textbf{formal,
  probabilistic grammar}from sequential data (e.g., sales
  conversations). A grammar is, by definition, an explanatory model
  because it defines the rules and structures that enable the generation
  of valid sequences. It provides an explicit model of the underlying
  communication structure. LLMs, on the other hand, do not learn
  explicit grammars in the classical sense, but rather statistical
  probabilities for token sequences.
\item
  \textbf{Transparency and traceability:}The ARS 2.0 methodology is
  transparent and comprehensible. The steps of data preparation, symbol
  assignment, grammar induction, simulation, and statistical validation
  are explicitly defined. The induced grammar itself is an interpretable
  result that serves as a hypothesis about the structure of
  communication. In contrast, the internal workings and decision-making
  of an LLM are opaque to the user.
\item
  \textbf{Hypothesis generation and testing:}ARS 2.0 works by generating
  hypotheses about the structure of interactions, which are then
  formalized using the induced grammar and statistically tested by
  comparison with empirical data (e.g., frequency distributions,
  correlation analyses). This corresponds to a scientific approach to
  explanation.
\item
  \textbf{Generative ability as an explanation:}The ability of the
  induced grammar to generate artificial sequences that are similar to
  the empirical data is an indication of its explanatory power. If the
  grammar can successfully reproduce the observed patterns, this
  indicates that it has "understood" the rules of dialogue---not in the
  human sense, but as a formal model.
\item
  \textbf{Qualitative and quantitative connection:}ARS 2.0 combines
  qualitative insights (e.g., categorization of conversational
  contributions) with quantitative methods (probabilistic rules,
  statistical tests) to create a robust and explanatory model.
\end{itemize}

\textbf{Conclusion:}

While LLM can impressively imitate dialogues without explaining the
underlying mechanisms, Algorithmic Recursive Sequence Analysis 2.0
offers a\textbf{explicitly explanatory model}in the form of a formal
grammar. This grammar reveals the rules according to which dialogues are
constructed and allows hypotheses about these structures to be generated
and statistically validated. In this sense, ARS 2.0 contributes directly
to\textbf{Understanding the structure and dynamics of dialogues}by
providing a transparent and testable explanatory model that goes beyond
mere imitation.

\end{document}
