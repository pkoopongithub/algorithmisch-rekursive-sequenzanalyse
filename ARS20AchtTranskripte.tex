% Options for packages loaded elsewhere
\PassOptionsToPackage{unicode}{hyperref}
\PassOptionsToPackage{hyphens}{url}
\documentclass[
]{article}
\usepackage{xcolor}
\usepackage{amsmath,amssymb}
\setcounter{secnumdepth}{-\maxdimen} % remove section numbering
\usepackage{iftex}
\ifPDFTeX
  \usepackage[T1]{fontenc}
  \usepackage[utf8]{inputenc}
  \usepackage{textcomp} % provide euro and other symbols
\else % if luatex or xetex
  \usepackage{unicode-math} % this also loads fontspec
  \defaultfontfeatures{Scale=MatchLowercase}
  \defaultfontfeatures[\rmfamily]{Ligatures=TeX,Scale=1}
\fi
\usepackage{lmodern}
\ifPDFTeX\else
  % xetex/luatex font selection
\fi
% Use upquote if available, for straight quotes in verbatim environments
\IfFileExists{upquote.sty}{\usepackage{upquote}}{}
\IfFileExists{microtype.sty}{% use microtype if available
  \usepackage[]{microtype}
  \UseMicrotypeSet[protrusion]{basicmath} % disable protrusion for tt fonts
}{}
\makeatletter
\@ifundefined{KOMAClassName}{% if non-KOMA class
  \IfFileExists{parskip.sty}{%
    \usepackage{parskip}
  }{% else
    \setlength{\parindent}{0pt}
    \setlength{\parskip}{6pt plus 2pt minus 1pt}}
}{% if KOMA class
  \KOMAoptions{parskip=half}}
\makeatother
\setlength{\emergencystretch}{3em} % prevent overfull lines
\providecommand{\tightlist}{%
  \setlength{\itemsep}{0pt}\setlength{\parskip}{0pt}}
\usepackage{bookmark}
\IfFileExists{xurl.sty}{\usepackage{xurl}}{} % add URL line breaks if available
\urlstyle{same}
\hypersetup{
  hidelinks,
  pdfcreator={LaTeX via pandoc}}

\author{}
\date{}

\begin{document}

\section{Algorithmisch Rekursive Sequenzanalyse
2.0}\label{algorithmisch-rekursive-sequenzanalyse-2.0}

\subsection{Grammatikinduktion aus weiteren Transkripten nach extensiver
Sequenzanalyse eines
Transkriptes}\label{grammatikinduktion-aus-weiteren-transkripten-nach-extensiver-sequenzanalyse-eines-transkriptes}

In diesem Dokument wird die Methode der algorithmisch rekursiven
Sequenzanalyse vorgestellt. Das Ziel dieser Methode ist es,
grammatikalische Strukturen aus natürlichen Sprachsequenzen zu
extrahieren und darauf aufbauend eine Grammatik zu induzieren. Wir
fokussieren uns auf die Analyse von Transkripten, die durch umfangreiche
Sequenzanalysen vorbereitet wurden, um die zugrundeliegenden Regeln und
Muster in den Daten zu erkennen.

\subsubsection{Hintergrund}\label{hintergrund}

Die Sequenzanalyse in natürlichen Sprachdaten ist ein weit verbreiteter
Ansatz in der Sprachverarbeitung, um wiederkehrende Muster, Strukturen
und Abhängigkeiten zwischen verschiedenen Teilen einer Sequenz zu
identifizieren. In dieser erweiterten Version (2.0) integrieren wir die
grammatikalische Induktion als eine Möglichkeit, aus den analysierten
Sequenzen eine formale Grammatik abzuleiten.

\subsubsection{Zielsetzung}\label{zielsetzung}

Die Hauptziele dieses Ansatzes umfassen:

\begin{enumerate}
\def\labelenumi{\arabic{enumi}.}
\item
  \textbf{Erweiterung der Sequenzanalyse}: Die Analyse wird auf eine
  breitere Menge an Transkripten angewendet, um zu prüfen, ob die
  entdeckten Muster über ein einzelnes Transkript hinaus generalisiert
  werden können.
\item
  \textbf{Grammatikinduktion}: Mit den gewonnenen Erkenntnissen aus den
  Sequenzen sollen grammatikalische Regeln extrahiert und formalisierte
  Grammatiken entwickelt werden.
\item
  \textbf{Evaluation der Qualität}: Die Qualität der induzierten
  Grammatik wird anhand ihrer Fähigkeit bewertet, zukünftige Sequenzen
  zu generieren, die den Mustern in den Transkripten entsprechen.
\end{enumerate}

\subsubsection{Methodik}\label{methodik}

Die Methode besteht aus mehreren Schritten, die iterativ durchgeführt
werden:

\begin{enumerate}
\def\labelenumi{\arabic{enumi}.}
\item
  \textbf{Datenvorbereitung}: Sammlung und Preprocessing der
  Transkripte.
\item
  \textbf{Sequenzanalyse}: Durchführung einer extensiven Sequenzanalyse
  mit Lesartenproduktion und Lesartenfalsifikation an einem Transkript
  (a.a.O in diesem Repository). Durchführung einer umfangreichen Analyse
  der Transkripte, bei der statistische und syntaktische Muster
  identifiziert werden.
\item
  \textbf{Grammatikinduktion}: Auf Basis der identifizierten Muster wird
  eine formale Grammatik (Python hier im Dokument) induziert, die die
  generierten Sequenzen beschreiben kann.
\item
  \textbf{Optimierung}: Die Grammatik wird optimiert (Python hier im
  Dokument) und validiert, um ihre Genauigkeit und Generalisierbarkeit
  zu erhöhen.
\end{enumerate}

Nach der Auswertung von Fachliteratur zu Verkaufsgesprächen ergibt sich
als vorläufige hypothetische Grammatik: Ein Verkaufsgespräch (VKG)
besteht aus einer Begrüßung (BG), einem Verkaufsteil (VT) und einer
Verabschiedung (AV).

\begin{itemize}
\item
  Die Begrüßung besteht aus einem Gruß des Kunden (KBG) und einem Gruß
  des Verkäufers (VBG).
\item
  Die Verabschiedung besteht aus einer Verabschiedung durch den Kunden
  (KAV) und einer Verabschiedung durch den Verkäufer (VAV).
\item
  Der Verkaufsteil besteht aus einem Bedarfsteil (B) und einem
  Abschlussteil (A).

  \begin{itemize}
  \item
    Der Bedarfsteil umfasst eine Bedarfsklärung (BBd) und eine
    Bedarfsargumentation (BA).

    \begin{itemize}
    \item
      Die Bedarfsklärung besteht aus Bedarfen des Kunden (KBBd) und den
      dazugehörigen Klärungen des Verkäufers (VBBd).
    \item
      Die Bedarfsargumentation setzt sich aus Argumenten des Kunden
      (KBA) und des Verkäufers (VBA) zusammen.
    \end{itemize}
  \item
    Der Abschlussteil (A) besteht aus Einwänden (AE) und einem
    Verkaufsabschluss (AA).

    \begin{itemize}
    \item
      Die Einwände setzen sich aus Argumenten des Kunden (KAE) und des
      Verkäufers (VAE) zusammen.
    \item
      Der Verkaufsabschluss besteht aus Argumenten des Kunden (KAA) und
      des Verkäufers (VAA).
    \end{itemize}
  \end{itemize}
\end{itemize}

Die Verabschiedung (AV) besteht aus einem Abschied des Kunden (KAV) und
des Verkäufers (VAV).

Das Startsymbol ist demnach VKG, und die Terminalzeichen für die
Kategorien sind: KBG, VBG, KBBd, VBBd, KBA, VBA, KAE, VAE, KAA, VAA, KAV
und VAV. ERstelle aus diesen Angaben eine probabilistische kontextfreie
Grammatik mit zunächst vermuteten Übergangswahrscheinlichkeiten.

ZUnächst wurde dann ein Transkript einer intensiven Sequenzanalyse mit
Lesartenproduktion und Lesartenfalsifikation, Kategorienbildung,
Interkodierkorrelation und Grammatik induktion mit Schem und
Grammatiktransduktion mit Lisp und Parsen der Terminalkette mit Object
Pascal durchgeführt (a.a.O. in diesem Repository).

Das Vorgehen lässt sich in mehrere klar definierte Schritte unterteilen.
Hier ist eine Übersicht, um sicherzustellen, dass alle Schritte
verständlich sind:

\subsubsection{\texorpdfstring{1. \textbf{Hypothesenbildung zur
Grammatik}}{1. Hypothesenbildung zur Grammatik}}\label{hypothesenbildung-zur-grammatik}

\begin{itemize}
\item
  Durchführung einer Voruntersuchung, um eine \textbf{hypothetische
  PCFG} zu erstellen. Diese dient als Orientierung, ist aber nicht
  verbindlich für die Analyse der Transkripte.
\end{itemize}

\subsubsection{\texorpdfstring{2. \textbf{Analyse der
Transkripte}}{2. Analyse der Transkripte}}\label{analyse-der-transkripte}

\begin{itemize}
\item
  Jedes \textbf{Transkript wird einzeln} analysiert. Die Intentionen der
  Sprecher und die Struktur der Interaktionen werden identifiziert.
\item
  Jeder Interaktion im Transkript wird ein \textbf{Terminalzeichen}
  zugeordnet. Dies ergibt für jedes Transkript eine
  \textbf{Terminalzeichenkette}.
\end{itemize}

\subsubsection{\texorpdfstring{3. \textbf{Induktion einer Grammatik für
jedes
Transkript}}{3. Induktion einer Grammatik für jedes Transkript}}\label{induktion-einer-grammatik-fuxfcr-jedes-transkript}

\begin{itemize}
\item
  Aus der Terminalzeichenkette eines jeden Transkripts wird eine
  spezifische \textbf{Grammatik induziert}.
\item
  Es gibt am Ende eine eigene \textbf{Grammatik pro Transkript}
  (insgesamt acht).
\end{itemize}

\subsubsection{\texorpdfstring{4. \textbf{Vereinigung der
Grammatiken}}{4. Vereinigung der Grammatiken}}\label{vereinigung-der-grammatiken}

\begin{itemize}
\item
  Die acht individuellen Grammatiken werden zu einer \textbf{vereinigten
  Grammatik} zusammengeführt, die die Struktur aller Transkripte
  abdeckt.
\end{itemize}

\subsubsection{\texorpdfstring{5. \textbf{Parsing der
Terminalzeichenketten}}{5. Parsing der Terminalzeichenketten}}\label{parsing-der-terminalzeichenketten}

\begin{itemize}
\item
  Die acht Terminalzeichenketten der Transkripte werden auf
  \textbf{Wohlgeformtheit} in Bezug auf die vereinigte Grammatik
  geprüft. Dies bedeutet, dass jede Zeichenkette durch die Grammatik
  korrekt erkannt und geparst werden sollte.
\end{itemize}

\subsubsection{\texorpdfstring{6. \textbf{Optimierung der
PCFG}}{6. Optimierung der PCFG}}\label{optimierung-der-pcfg}

\begin{itemize}
\item
  Eine erste Version einer PCFG wird aus den vorherigen Analysen
  erstellt.
\item
  Mit der erstellten PCFG werden \textbf{künstliche
  Terminalzeichenketten} erzeugt.
\item
  Die \textbf{anteilige Häufigkeit der Terminalzeichen} aus den
  künstlichen Zeichenketten wird mit der Häufigkeit der Terminalzeichen
  in den realen Daten (den acht Transkripten) verglichen.
\item
  Die \textbf{Signifikanz} dieser Korrelation wird gemessen.
\item
  Anpassungen an der PCFG werden vorgenommen, bis die Korrelation
  zufriedenstellend ist.
\end{itemize}

\subsubsection{\texorpdfstring{7.
\textbf{Wiederholungsprozess}}{7. Wiederholungsprozess}}\label{wiederholungsprozess}

\begin{itemize}
\item
  Schritte 6 und 7 werden so lange wiederholt, bis eine gute Anpassung
  zwischen den \textbf{empirischen Daten} und der erzeugten PCFG
  besteht.
\end{itemize}

Hier ist eine probabilistische kontextfreie Grammatik (PCFG), die auf
der Struktur basiert. Bei einer PCFG werden Wahrscheinlichkeiten für die
Übergänge zwischen verschiedenen Produktionsregeln definiert. Da du nur
eine Struktur ohne spezifische Wahrscheinlichkeiten angegeben hast,
werde ich annahmeweise gleiche Wahrscheinlichkeiten für die möglichen
Optionen innerhalb der gleichen Ebene verwenden, um eine Grundlage zu
schaffen. Diese Wahrscheinlichkeiten lassen sich später anpassen, wenn
empirische Daten verfügbar sind.

\section{Probabilistische hypothetische kontextfreie Grammatik (PCFG)
für ein
Verkaufsgespräch}\label{probabilistische-hypothetische-kontextfreie-grammatik-pcfg-fuxfcr-ein-verkaufsgespruxe4ch}

VKG -\textgreater{} BG VT AV {[}1.0{]}

\section{Begrüßung}\label{begruxfcuxdfung}

BG -\textgreater{} KBG VBG {[}1.0{]} KBG -\textgreater{}
\textquotesingle Kunden-Gruß\textquotesingle{} VBG -\textgreater{}
\textquotesingle Verkäufer-Gruß\textquotesingle{}

\section{Verabschiedung}\label{verabschiedung}

AV -\textgreater{} KAV VAV {[}1.0{]} KAV -\textgreater{}
\textquotesingle Kunden-Verabschiedung\textquotesingle{} VAV
-\textgreater{}
\textquotesingle Verkäufer-Verabschiedung\textquotesingle{}

\section{Verkaufsteil}\label{verkaufsteil}

VT -\textgreater{} B A {[}1.0{]}

\section{Bedarfsteil}\label{bedarfsteil}

B -\textgreater{} BBd BA {[}1.0{]}

\section{Bedarfsklärung}\label{bedarfskluxe4rung}

BBd -\textgreater{} KBBd VBBd {[}1.0{]} KBBd -\textgreater{}
\textquotesingle Kunden-Bedarf\textquotesingle{} VBBd -\textgreater{}
\textquotesingle Verkäufer-Klärung\textquotesingle{}

\section{Bedarfsargumentation}\label{bedarfsargumentation}

BA -\textgreater{} KBA VBA {[}0.5{]} \textbar{} VBA KBA {[}0.5{]} KBA
-\textgreater{} \textquotesingle Kunden-Argument\textquotesingle{} VBA
-\textgreater{} \textquotesingle Verkäufer-Argument\textquotesingle{}

\section{Abschlussteil}\label{abschlussteil}

A -\textgreater{} AE AA {[}1.0{]}

\section{Einwände}\label{einwuxe4nde}

AE -\textgreater{} KAE VAE {[}0.5{]} \textbar{} VAE KAE {[}0.5{]} KAE
-\textgreater{} \textquotesingle Kunden-Einwand\textquotesingle{} VAE
-\textgreater{} \textquotesingle Verkäufer-Einwand\textquotesingle{}

\section{Verkaufsabschluss}\label{verkaufsabschluss}

AA -\textgreater{} KAA VAA {[}0.5{]} \textbar{} VAA KAA {[}0.5{]} KAA
-\textgreater{} \textquotesingle Kunden-Abschluss\textquotesingle{} VAA
-\textgreater{} \textquotesingle Verkäufer-Abschluss\textquotesingle{}

Es folgen die acht Transkripte. (Tonbandprotokolle a.a.O. in diesem
Repository.

\subsubsection{\texorpdfstring{\textbf{Text 1}}{Text 1}}\label{text-1}

\textbf{Datum:} 28. Juni 1994, \textbf{Ort:} Metzgerei, Aachen, 11:00
Uhr

\emph{(Die Geräusche eines geschäftigen Marktplatzes im Hintergrund,
Stimmen und Gemurmel)}

\textbf{Verkäuferin:} Guten Tag, was darf es sein?

\textbf{Kunde:} Einmal von der groben Leberwurst, bitte.

\textbf{Verkäuferin:} Wie viel darf's denn sein?

\textbf{Kunde:} Zwei hundert Gramm.

\textbf{Verkäuferin:} Zwei hundert Gramm. Sonst noch etwas?

\textbf{Kunde:} Ja, dann noch ein Stück von dem Schwarzwälder Schinken.

\textbf{Verkäuferin:} Wie groß soll das Stück sein?

\textbf{Kunde:} So um die dreihundert Gramm.

\textbf{Verkäuferin:} Alles klar. Kommt sofort. \emph{(Geräusche von
Papier und Verpackung)}

\textbf{Kunde:} Danke schön.

\textbf{Verkäuferin:} Das macht dann acht Mark zwanzig.

\textbf{Kunde:} Bitte. \emph{(Klimpern von Münzen, Geräusche der Kasse)}

\textbf{Verkäuferin:} Danke und einen schönen Tag noch!

\textbf{Kunde:} Danke, ebenfalls!

\textbf{Ende Text 1}

\subsubsection{\texorpdfstring{\textbf{Text 2}}{Text 2}}\label{text-2}

\textbf{Datum:} 28. Juni 1994, \textbf{Ort:} Marktplatz, Aachen

\emph{(Ständige Hintergrundgeräusche von Stimmen und Marktatmosphäre)}

\textbf{Verkäufer:} Kirschen kann jeder probieren hier, Kirschen kann
jeder probieren hier!

\textbf{Kunde 1:} Ein halbes Kilo Kirschen, bitte.

\textbf{Verkäufer:} Ein halbes Kilo? Oder ein Kilo?

\emph{(Unverständliches Gespräch, Münzen klimpern)}

\textbf{Verkäufer:} Danke schön!

\textbf{Verkäufer:} Kirschen kann jeder probieren hier! Drei Mark,
bitte.

\textbf{Kunde 1:} Danke schön!

\textbf{Verkäufer:} Kirschen kann jeder probieren hier, Kirschen kann
jeder probieren hier!

\emph{(Weitere Stimmen im Hintergrund, unverständliches Gespräch, Münzen
klimpern)}

\textbf{Kunde 2:} Ein halbes Kilo, bitte.

\emph{(Unverständliches Gespräch)}

\textbf{Ende Text 2}

\subsubsection{\texorpdfstring{\textbf{Text 3}}{Text 3}}\label{text-3}

\textbf{Datum:} 28. Juni 1994, \textbf{Ort:} Fischstand, Marktplatz,
Aachen

\emph{(Marktatmosphäre, Gespräch im Hintergrund, teilweise
unverständlich)}

\textbf{Kunde:} Ein Pfund Seelachs, bitte.

\textbf{Verkäufer:} Seelachs, alles klar.

\emph{(Geräusche von Verpackung und Verkaufsvorbereitungen)}

\textbf{Verkäufer:} Vier Mark neunzehn, bitte.

\emph{(Geräusche von Verpackung, Münzen klimpern)}

\textbf{Verkäufer:} Schönen Dank!

\textbf{Kunde:} Ja, danke schön!

\textbf{Ende Text 3}

\subsubsection{\texorpdfstring{\textbf{Text 4}}{Text 4}}\label{text-4}

\textbf{Datum:} 28. Juni 1994, \textbf{Ort:} Gemüsestand, Aachen,
Marktplatz, 11:00 Uhr

\emph{(Marktatmosphäre, teilweise unverständlich)}

\textbf{Kunde:} Hören Sie, ich nehme ein paar Champignons mit.

\textbf{Verkäufer:} Braune oder helle?

\textbf{Kunde:} Nehmen wir die hellen.

\textbf{Verkäufer:} Alles klar, die hellen.

\emph{(Unverständliche Unterhaltung im Hintergrund)}

\textbf{Verkäufer:} Die sind beide frisch, keine Sorge.

\textbf{Kunde:} Wie ist es mit Pfifferlingen?

\textbf{Verkäufer:} Ah, die sind super!

\emph{(Unverständliches Gespräch)}

\textbf{Kunde:} Kann ich die in Reissalat tun?

\textbf{Verkäufer:} Eher kurz anbraten in der Pfanne.

\textbf{Kunde:} Okay, mache ich.

\textbf{Verkäufer:} Die können Sie roh verwenden, aber ein bisschen
anbraten ist besser.

\textbf{Kunde:} Verstanden.

\emph{(Weitere Unterhaltung, unverständliche Kommentare)}

\textbf{Verkäufer:} Noch etwas anderes?

\textbf{Kunde:} Ja, dann nehme ich noch Erdbeeren.

\emph{(Pause, Hintergrundgeräusche von Verpackung und Stimmen)}

\textbf{Verkäufer:} Schönen Tag noch!

\textbf{Kunde:} Gleichfalls!

\textbf{Ende Text 4}

\subsubsection{\texorpdfstring{\textbf{Text 5}}{Text 5}}\label{text-5}

\textbf{Datum:} 26. Juni 1994, \textbf{Ort:} Gemüsestand, Aachen,
Marktplatz, 11:00 Uhr

\emph{(Marktatmosphäre, teilweise unverständlich)}

\textbf{Verkäufer:} So, bitte schön.

\textbf{Kunde 1:} Auf Wiedersehen!

\textbf{Kunde 2:} Ich hätte gern ein Kilo von den Granny Smith Äpfeln
hier.

\emph{(Unverständliches Gespräch im Hintergrund)}

\textbf{Verkäufer:} Sonst noch etwas?

\textbf{Kunde 2:} Ja, noch ein Kilo Zwiebeln.

\textbf{Verkäufer:} Alles klar.

\emph{(Unverständliches Gespräch, Hintergrundgeräusche)}

\textbf{Kunde 2:} Das war\textquotesingle s.

\textbf{Verkäufer:} Sechs Mark fünfundzwanzig, bitte.

\emph{(Unverständliches Gespräch, Geräusche von Münzen und Verpackung)}

\textbf{Verkäufer:} Wiedersehen!

\textbf{Kunde 2:} Wiedersehen!

\textbf{Ende Text 5}

\subsubsection{\texorpdfstring{\textbf{Text 6}}{Text 6}}\label{text-6}

\textbf{Datum:} 28. Juni 1994, \textbf{Ort:} Käseverkaufsstand, Aachen,
Marktplatz

\emph{(Marktatmosphäre, Begrüßungen)}

\textbf{Kunde 1:} Guten Morgen!

\textbf{Verkäufer:} Guten Morgen!

\textbf{Kunde 1:} Ich hätte gerne fünfhundert Gramm holländischen Gouda.

\textbf{Verkäufer:} Am Stück?

\textbf{Kunde 1:} Ja, am Stück, bitte.

\textbf{Ende Text 6}

\subsubsection{\texorpdfstring{\textbf{Text 7}}{Text 7}}\label{text-7}

\textbf{Datum:} 28. Juni 1994, \textbf{Ort:} Bonbonstand, Aachen,
Marktplatz, 11:30 Uhr

\emph{(Geräusche von Stimmen und Marktatmosphäre, teilweise
unverständlich)}

\textbf{Kunde:} Von den gemischten hätte ich gerne hundert Gramm.

\emph{(Unverständliche Fragen und Antworten)}

\textbf{Verkäufer:} Für zu Hause oder zum Mitnehmen?

\textbf{Kunde:} Zum Mitnehmen, bitte.

\textbf{Verkäufer:} Fünfzig Pfennig, bitte.

\emph{(Klimpern von Münzen, Geräusche von Verpackung)}

\textbf{Kunde:} Danke!

\textbf{Ende Text 7}

\subsubsection{\texorpdfstring{\textbf{Text 8}}{Text 8}}\label{text-8}

\textbf{Datum:} 9. Juli 1994, \textbf{Ort:} Bäckerei, Aachen, 12:00 Uhr

\emph{(Schritte hörbar, Hintergrundgeräusche, teilweise unverständlich)}

\textbf{Kunde:} Guten Tag!

\emph{(Unverständliche Begrüßung im Hintergrund)}

\textbf{Verkäuferin:} Einmal unser bester Kaffee, frisch gemahlen,
bitte.

\emph{(Geräusche der Kaffeemühle, Verpackungsgeräusche)}

\textbf{Verkäuferin:} Sonst noch etwas?

\textbf{Kunde:} Ja, noch zwei Stück Obstsalat und ein Schälchen Sahne.

\textbf{Verkäuferin:} In Ordnung!

\emph{(Geräusche der Kaffeemühle, Papiergeräusche)}

\textbf{Verkäuferin:} Ein kleines Schälchen Sahne, ja?

\textbf{Kunde:} Ja, danke.

\emph{(Türgeräusch, Lachen, Papiergeräusche)}

\textbf{Verkäuferin:} Keiner kümmert sich darum, die Türen zu ölen.

\textbf{Kunde:} Ja, das ist immer so.

\emph{(Lachen, Geräusche von Münzen und Verpackung)}

\textbf{Verkäuferin:} Das macht vierzehn Mark und neunzehn Pfennig,
bitte.

\textbf{Kunde:} Ich zahle in Kleingeld.

\emph{(Lachen und Geräusche von Münzen)}

\textbf{Verkäuferin:} Vielen Dank, schönen Sonntag noch!

\textbf{Kunde:} Danke, Ihnen auch!

\textbf{Ende Text 8}

Auf Grundlage der vorläufigen probabilistischen kontextfreien Grammatik
(PCFG) werden die Transkripte in die entsprechenden Terminalzeichen
umwandeln. Hier ist die detaillierte Zuordnung für jedes Transkript,
wobei für jede relevante Aktion oder Aussage des Gesprächs ein passendes
Terminalzeichen aus der vorgegebenen Grammatik verwende.

\subsubsection{\texorpdfstring{\textbf{Transkript 1 -
Terminalzeichen}}{Transkript 1 - Terminalzeichen}}\label{transkript-1---terminalzeichen}

\textbf{Datum:} 28. Juni 1994, \textbf{Ort:} Metzgerei, Aachen, 11:00
Uhr\textbar{} Kunde: Guten Tag \textbar{} KBG (Kunden-Gruß) \textbar{}
Verkäuferin: Guten Tag \textbar{} VBG (Verkäufer-Gruß) \textbar{} Kunde:
Einmal von der groben Leberwurst, bitte. \textbar{} KBBd (Kunden-Bedarf)
\textbar{} Verkäuferin: Wie viel darf's denn sein? \textbar{} VBBd
(Verkäufer-Klärung) \textbar{} Kunde: Zwei hundert Gramm. \textbar{} KBA
(Kunden-Argument) \textbar{} Verkäuferin: Sonst noch etwas? \textbar{}
VBA (Verkäufer-Argument) \textbar{} Kunde: Ja, dann noch ein Stück von
dem Schwarzwälde\textbar{} KBBd (Kunden-Bedarf) \textbar{} Verkäuferin:
Wie groß soll das Stück sein? \textbar{} VBBd (Verkäufer-Klärung)
\textbar{} Kunde: So um die dreihundert Gramm. \textbar{} KBA
(Kunden-Argument) \textbar{} Verkäuferin: Das macht dann acht Mark
zwanzig. \textbar{} VAA (Verkäufer-Abschluss) \textbar{} Kunde: Bitte.
\textbar{} KAA (Kunden-Abschluss) \textbar{} Verkäuferin: Danke und
einen schönen Tag noch! \textbar{} VAV (Verkäufer-Verabschiedung)
\textbar{} Kunde: Danke, ebenfalls! \textbar{} KAV
(Kunden-Verabschiedung)\textbar{}

\subsubsection{\texorpdfstring{\textbf{Transkript 2 -
Terminalzeichen}}{Transkript 2 - Terminalzeichen}}\label{transkript-2---terminalzeichen}

\textbf{Datum:} 28. Juni 1994, \textbf{Ort:} Marktplatz,
Aachen\textbar{} Verkäufer: Kirschen kann jeder probieren hier!
\textbar{} VBG (Verkäufer-Gruß) \textbar{} Kunde 1: Ein halbes Kilo
Kirschen, bitte. \textbar{} KBBd (Kunden-Bedarf) \textbar{} Verkäufer:
Ein halbes Kilo? Oder ein Kilo? \textbar{} VBBd (Verkäufer-Klärung)
\textbar{} Verkäufer: Drei Mark, bitte. \textbar{} VAA
(Verkäufer-Abschluss) \textbar{} Kunde 1: Danke schön! \textbar{} KAA
(Kunden-Abschluss) \textbar{} Verkäufer: Kirschen kann jeder probieren
hier! \textbar{} VBG (Verkäufer-Gruß) \textbar{} Kunde 2: Ein halbes
Kilo, bitte. \textbar{} KBBd (Kunden-Bedarf) \textbar{} Verkäufer: Drei
Mark, bitte. \textbar{} VAA (Verkäufer-Abschluss) \textbar{} Kunde 2:
Danke schön! \textbar{} KAA (Kunden-Abschluss)

\subsubsection{\texorpdfstring{\textbf{Transkript 3 -
Terminalzeichen}}{Transkript 3 - Terminalzeichen}}\label{transkript-3---terminalzeichen}

\textbf{Datum:} 28. Juni 1994, \textbf{Ort:} Fischstand, Marktplatz,
Aachen\textbar{} Kunde: Ein Pfund Seelachs, bitte. \textbar{} KBBd
(Kunden-Bedarf) \textbar{} Verkäufer: Seelachs, alles klar. \textbar{}
VBBd (Verkäufer-Klärung) \textbar{} Verkäufer: Vier Mark neunzehn,
bitte. \textbar{} VAA (Verkäufer-Abschluss) \textbar{} Kunde: Danke
schön! \textbar{} KAA (Kunden-Abschluss)

\subsubsection{\texorpdfstring{\textbf{Transkript 4 -
Terminalzeichen}}{Transkript 4 - Terminalzeichen}}\label{transkript-4---terminalzeichen}

\textbf{Datum:} 28. Juni 1994, \textbf{Ort:} Gemüsestand, Aachen,
Marktplatz, 11:00 Uhr\textbar{} Kunde: Hören Sie, ich nehme ein paar
Champignons mit. \textbar{} KBBd (Kunden-Bedarf) \textbar{} Verkäufer:
Braune oder helle? \textbar{} VBBd (Verkäufer-Klärung) \textbar{} Kunde:
Nehmen wir die hellen. \textbar{} KBA (Kunden-Argument) \textbar{}
Verkäufer: Die sind beide frisch, keine Sorge. \textbar{} VBA
(Verkäufer-Argument) \textbar{} Kunde: Wie ist es mit Pfifferlingen?
\textbar{} KBBd (Kunden-Bedarf) \textbar{} Verkäufer: Ah, die sind
super! \textbar{} VBA (Verkäufer-Argument) \textbar{} Kunde: Kann ich
die in Reissalat tun? \textbar{} KAE (Kunden-Einwand) \textbar{}
Verkäufer: Eher kurz anbraten in der Pfanne. \textbar{} VAE
(Verkäufer-Einwand) \textbar{} Kunde: Okay, mache ich. \textbar{} KAA
(Kunden-Abschluss) \textbar{} Verkäufer: Schönen Tag noch! \textbar{}
VAV (Verkäufer-Verabschiedung) \textbar{} Kunde: Gleichfalls! \textbar{}
KAV (Kunden-Verabschiedung)

\subsubsection{\texorpdfstring{\textbf{Transkript 5 -
Terminalzeichen}}{Transkript 5 - Terminalzeichen}}\label{transkript-5---terminalzeichen}

\textbf{Datum:} 26. Juni 1994, \textbf{Ort:} Gemüsestand, Aachen,
Marktplatz, 11:00 Uhr\textbar{} Kunde 1: Auf Wiedersehen! \textbar{} KAV
(Kunden-Verabschiedung) \textbar{} Kunde 2: Ich hätte gern ein Kilo von
den Granny Smi\textbar{} KBBd (Kunden-Bedarf) \textbar{} Verkäufer:
Sonst noch etwas? \textbar{} VBBd (Verkäufer-Klärung) \textbar{} Kunde
2: Ja, noch ein Kilo Zwiebeln. \textbar{} KBBd (Kunden-Bedarf)
\textbar{} Verkäufer: Sechs Mark fünfundzwanzig, bitte. \textbar{} VAA
(Verkäufer-Abschluss) \textbar{} Kunde 2: Auf Wiedersehen! \textbar{}
KAV (Kunden-Verabschiedung)

\subsubsection{\texorpdfstring{\textbf{Transkript 6 -
Terminalzeichen}}{Transkript 6 - Terminalzeichen}}\label{transkript-6---terminalzeichen}

\textbf{Datum:} 28. Juni 1994, \textbf{Ort:} Käseverkaufsstand, Aachen,
Marktplatz\textbar{} Kunde 1: Guten Morgen! \textbar{} KBG (Kunden-Gruß)
\textbar{} Verkäufer: Guten Morgen! \textbar{} VBG (Verkäufer-Gruß)
\textbar{} Kunde 1: Ich hätte gerne fünfhundert Gramm holländi\textbar{}
KBBd (Kunden-Bedarf) \textbar{} Verkäufer: Am Stück? \textbar{} VBBd
(Verkäufer-Klärung) \textbar{} Kunde 1: Ja, am Stück, bitte. \textbar{}
KAA (Kunden-Abschluss)

\subsubsection{\texorpdfstring{\textbf{Transkript 7 -
Terminalzeichen}}{Transkript 7 - Terminalzeichen}}\label{transkript-7---terminalzeichen}

\textbf{Datum:} 28. Juni 1994, \textbf{Ort:} Bonbonstand, Aachen,
Marktplatz, 11:30 Uhr\textbar{} Kunde: Von den gemischten hätte ich
gerne hundert G\textbar{} KBBd (Kunden-Bedarf) \textbar{} Verkäufer: Für
zu Hause oder zum Mitnehmen? \textbar{} VBBd (Verkäufer-Klärung)
\textbar{} Kunde: Zum Mitnehmen, bitte. \textbar{} KBA (Kunden-Argument)
\textbar{} Verkäufer: Fünfzig Pfennig, bitte. \textbar{} VAA
(Verkäufer-Abschluss) \textbar{} Kunde: Danke! \textbar{} KAA
(Kunden-Abschluss)

\subsubsection{\texorpdfstring{\textbf{Transkript 8 -
Terminalzeichen}}{Transkript 8 - Terminalzeichen}}\label{transkript-8---terminalzeichen}

\textbf{Datum:} 9. Juli 1994, \textbf{Ort:} Bäckerei, Aachen, 12:00
Uhr\textbar{} Kunde: Guten Tag! \textbar{} KBG (Kunden-Gruß) \textbar{}
Verkäuferin: Einmal unser bester Kaffee, frisch gem\textbar{} VBBd
(Verkäufer-Klärung) \textbar{} Kunde: Ja, noch zwei Stück Obstsalat und
ein Schälc\textbar{} KBBd (Kunden-Bedarf) \textbar{} Verkäuferin: In
Ordnung! \textbar{} VBA (Verkäufer-Argument) \textbar{} Verkäuferin: Das
macht vierzehn Mark und neunzehn P\textbar{} VAA (Verkäufer-Abschluss)
\textbar{} Kunde: Ich zahle in Kleingeld. \textbar{} KAA
(Kunden-Abschluss) \textbar{} Verkäuferin: Vielen Dank, schönen Sonntag
noch! \textbar{} VAV (Verkäufer-Verabschiedung) \textbar{} Kunde: Danke,
Ihnen auch! \textbar{} KAV (Kunden-Verabschiedung

Dann folg auf der Basis dieser acht Terminalzeichenketten die Induktion
und Optimierung einer auf dem gesamten erhobenen empirischen Material
beruhende Induktion einer probabilistischen Grammatik und ihre
OPtimierung.

Auf Grundlage der vorläufigen probabilistischen kontextfreien Grammatik
(PCFG) werden die Transkripte in die entsprechenden Terminalzeichen
umwandeln. Hier ist die detaillierte Zuordnung für jedes Transkript,
wobei für jede relevante Aktion oder Aussage des Gesprächs ein passendes
Terminalzeichen aus der vorgegebenen Grammatik verwende.

\subsubsection{\texorpdfstring{\textbf{Transkript 1 -
Terminalzeichen}}{Transkript 1 - Terminalzeichen}}\label{transkript-1---terminalzeichen-1}

\textbf{Datum:} 28. Juni 1994, \textbf{Ort:} Metzgerei, Aachen, 11:00
Uhr\textbar{} Kunde: Guten Tag \textbar{} KBG (Kunden-Gruß) \textbar{}
Verkäuferin: Guten Tag \textbar{} VBG (Verkäufer-Gruß) \textbar{} Kunde:
Einmal von der groben Leberwurst, bitte. \textbar{} KBBd (Kunden-Bedarf)
\textbar{} Verkäuferin: Wie viel darf's denn sein? \textbar{} VBBd
(Verkäufer-Klärung) \textbar{} Kunde: Zwei hundert Gramm. \textbar{} KBA
(Kunden-Argument) \textbar{} Verkäuferin: Sonst noch etwas? \textbar{}
VBA (Verkäufer-Argument) \textbar{} Kunde: Ja, dann noch ein Stück von
dem Schwarzwälde\textbar{} KBBd (Kunden-Bedarf) \textbar{} Verkäuferin:
Wie groß soll das Stück sein? \textbar{} VBBd (Verkäufer-Klärung)
\textbar{} Kunde: So um die dreihundert Gramm. \textbar{} KBA
(Kunden-Argument) \textbar{} Verkäuferin: Das macht dann acht Mark
zwanzig. \textbar{} VAA (Verkäufer-Abschluss) \textbar{} Kunde: Bitte.
\textbar{} KAA (Kunden-Abschluss) \textbar{} Verkäuferin: Danke und
einen schönen Tag noch! \textbar{} VAV (Verkäufer-Verabschiedung)
\textbar{} Kunde: Danke, ebenfalls! \textbar{} KAV
(Kunden-Verabschiedung)\textbar{}

\subsubsection{\texorpdfstring{\textbf{Transkript 2 -
Terminalzeichen}}{Transkript 2 - Terminalzeichen}}\label{transkript-2---terminalzeichen-1}

\textbf{Datum:} 28. Juni 1994, \textbf{Ort:} Marktplatz,
Aachen\textbar{} Verkäufer: Kirschen kann jeder probieren hier!
\textbar{} VBG (Verkäufer-Gruß) \textbar{} Kunde 1: Ein halbes Kilo
Kirschen, bitte. \textbar{} KBBd (Kunden-Bedarf) \textbar{} Verkäufer:
Ein halbes Kilo? Oder ein Kilo? \textbar{} VBBd (Verkäufer-Klärung)
\textbar{} Verkäufer: Drei Mark, bitte. \textbar{} VAA
(Verkäufer-Abschluss) \textbar{} Kunde 1: Danke schön! \textbar{} KAA
(Kunden-Abschluss) \textbar{} Verkäufer: Kirschen kann jeder probieren
hier! \textbar{} VBG (Verkäufer-Gruß) \textbar{} Kunde 2: Ein halbes
Kilo, bitte. \textbar{} KBBd (Kunden-Bedarf) \textbar{} Verkäufer: Drei
Mark, bitte. \textbar{} VAA (Verkäufer-Abschluss) \textbar{} Kunde 2:
Danke schön! \textbar{} KAA (Kunden-Abschluss)

\subsubsection{\texorpdfstring{\textbf{Transkript 3 -
Terminalzeichen}}{Transkript 3 - Terminalzeichen}}\label{transkript-3---terminalzeichen-1}

\textbf{Datum:} 28. Juni 1994, \textbf{Ort:} Fischstand, Marktplatz,
Aachen\textbar{} Kunde: Ein Pfund Seelachs, bitte. \textbar{} KBBd
(Kunden-Bedarf) \textbar{} Verkäufer: Seelachs, alles klar. \textbar{}
VBBd (Verkäufer-Klärung) \textbar{} Verkäufer: Vier Mark neunzehn,
bitte. \textbar{} VAA (Verkäufer-Abschluss) \textbar{} Kunde: Danke
schön! \textbar{} KAA (Kunden-Abschluss)

\subsubsection{\texorpdfstring{\textbf{Transkript 4 -
Terminalzeichen}}{Transkript 4 - Terminalzeichen}}\label{transkript-4---terminalzeichen-1}

\textbf{Datum:} 28. Juni 1994, \textbf{Ort:} Gemüsestand, Aachen,
Marktplatz, 11:00 Uhr\textbar{} Kunde: Hören Sie, ich nehme ein paar
Champignons mit. \textbar{} KBBd (Kunden-Bedarf) \textbar{} Verkäufer:
Braune oder helle? \textbar{} VBBd (Verkäufer-Klärung) \textbar{} Kunde:
Nehmen wir die hellen. \textbar{} KBA (Kunden-Argument) \textbar{}
Verkäufer: Die sind beide frisch, keine Sorge. \textbar{} VBA
(Verkäufer-Argument) \textbar{} Kunde: Wie ist es mit Pfifferlingen?
\textbar{} KBBd (Kunden-Bedarf) \textbar{} Verkäufer: Ah, die sind
super! \textbar{} VBA (Verkäufer-Argument) \textbar{} Kunde: Kann ich
die in Reissalat tun? \textbar{} KAE (Kunden-Einwand) \textbar{}
Verkäufer: Eher kurz anbraten in der Pfanne. \textbar{} VAE
(Verkäufer-Einwand) \textbar{} Kunde: Okay, mache ich. \textbar{} KAA
(Kunden-Abschluss) \textbar{} Verkäufer: Schönen Tag noch! \textbar{}
VAV (Verkäufer-Verabschiedung) \textbar{} Kunde: Gleichfalls! \textbar{}
KAV (Kunden-Verabschiedung)

\subsubsection{\texorpdfstring{\textbf{Transkript 5 -
Terminalzeichen}}{Transkript 5 - Terminalzeichen}}\label{transkript-5---terminalzeichen-1}

\textbf{Datum:} 26. Juni 1994, \textbf{Ort:} Gemüsestand, Aachen,
Marktplatz, 11:00 Uhr\textbar{} Kunde 1: Auf Wiedersehen! \textbar{} KAV
(Kunden-Verabschiedung) \textbar{} Kunde 2: Ich hätte gern ein Kilo von
den Granny Smi\textbar{} KBBd (Kunden-Bedarf) \textbar{} Verkäufer:
Sonst noch etwas? \textbar{} VBBd (Verkäufer-Klärung) \textbar{} Kunde
2: Ja, noch ein Kilo Zwiebeln. \textbar{} KBBd (Kunden-Bedarf)
\textbar{} Verkäufer: Sechs Mark fünfundzwanzig, bitte. \textbar{} VAA
(Verkäufer-Abschluss) \textbar{} Kunde 2: Auf Wiedersehen! \textbar{}
KAV (Kunden-Verabschiedung)

\subsubsection{\texorpdfstring{\textbf{Transkript 6 -
Terminalzeichen}}{Transkript 6 - Terminalzeichen}}\label{transkript-6---terminalzeichen-1}

\textbf{Datum:} 28. Juni 1994, \textbf{Ort:} Käseverkaufsstand, Aachen,
Marktplatz\textbar{} Kunde 1: Guten Morgen! \textbar{} KBG (Kunden-Gruß)
\textbar{} Verkäufer: Guten Morgen! \textbar{} VBG (Verkäufer-Gruß)
\textbar{} Kunde 1: Ich hätte gerne fünfhundert Gramm holländi\textbar{}
KBBd (Kunden-Bedarf) \textbar{} Verkäufer: Am Stück? \textbar{} VBBd
(Verkäufer-Klärung) \textbar{} Kunde 1: Ja, am Stück, bitte. \textbar{}
KAA (Kunden-Abschluss)

\subsubsection{\texorpdfstring{\textbf{Transkript 7 -
Terminalzeichen}}{Transkript 7 - Terminalzeichen}}\label{transkript-7---terminalzeichen-1}

\textbf{Datum:} 28. Juni 1994, \textbf{Ort:} Bonbonstand, Aachen,
Marktplatz, 11:30 Uhr\textbar{} Kunde: Von den gemischten hätte ich
gerne hundert G\textbar{} KBBd (Kunden-Bedarf) \textbar{} Verkäufer: Für
zu Hause oder zum Mitnehmen? \textbar{} VBBd (Verkäufer-Klärung)
\textbar{} Kunde: Zum Mitnehmen, bitte. \textbar{} KBA (Kunden-Argument)
\textbar{} Verkäufer: Fünfzig Pfennig, bitte. \textbar{} VAA
(Verkäufer-Abschluss) \textbar{} Kunde: Danke! \textbar{} KAA
(Kunden-Abschluss)

\subsubsection{\texorpdfstring{\textbf{Transkript 8 -
Terminalzeichen}}{Transkript 8 - Terminalzeichen}}\label{transkript-8---terminalzeichen-1}

\textbf{Datum:} 9. Juli 1994, \textbf{Ort:} Bäckerei, Aachen, 12:00
Uhr\textbar{} Kunde: Guten Tag! \textbar{} KBG (Kunden-Gruß) \textbar{}
Verkäuferin: Einmal unser bester Kaffee, frisch gem\textbar{} VBBd
(Verkäufer-Klärung) \textbar{} Kunde: Ja, noch zwei Stück Obstsalat und
ein Schälc\textbar{} KBBd (Kunden-Bedarf) \textbar{} Verkäuferin: In
Ordnung! \textbar{} VBA (Verkäufer-Argument) \textbar{} Verkäuferin: Das
macht vierzehn Mark und neunzehn P\textbar{} VAA (Verkäufer-Abschluss)
\textbar{} Kunde: Ich zahle in Kleingeld. \textbar{} KAA
(Kunden-Abschluss) \textbar{} Verkäuferin: Vielen Dank, schönen Sonntag
noch! \textbar{} VAV (Verkäufer-Verabschiedung) \textbar{} Kunde: Danke,
Ihnen auch! \textbar{} KAV (Kunden-Verabschiedung

\textbf{import} numpy \textbf{as} np

\textbf{from} scipy.stats \textbf{import} pearsonr

\emph{\# Neue empirische Terminalzeichenketten}

empirical\_chains \textbf{=} {[}

{[}\textquotesingle KBG\textquotesingle,
\textquotesingle VBG\textquotesingle,
\textquotesingle KBBd\textquotesingle,
\textquotesingle VBBd\textquotesingle,
\textquotesingle KBA\textquotesingle,
\textquotesingle VBA\textquotesingle,
\textquotesingle KBBd\textquotesingle,
\textquotesingle VBBd\textquotesingle,
\textquotesingle KBA\textquotesingle,
\textquotesingle VAA\textquotesingle,
\textquotesingle KAA\textquotesingle,
\textquotesingle VAV\textquotesingle,
\textquotesingle KAV\textquotesingle{]},

{[}\textquotesingle VBG\textquotesingle,
\textquotesingle KBBd\textquotesingle,
\textquotesingle VBBd\textquotesingle,
\textquotesingle VAA\textquotesingle,
\textquotesingle KAA\textquotesingle,
\textquotesingle VBG\textquotesingle,
\textquotesingle KBBd\textquotesingle,
\textquotesingle VAA\textquotesingle,
\textquotesingle KAA\textquotesingle{]},

{[}\textquotesingle KBBd\textquotesingle,
\textquotesingle VBBd\textquotesingle,
\textquotesingle VAA\textquotesingle,
\textquotesingle KAA\textquotesingle{]},

{[}\textquotesingle KBBd\textquotesingle,
\textquotesingle VBBd\textquotesingle,
\textquotesingle KBA\textquotesingle,
\textquotesingle VBA\textquotesingle,
\textquotesingle KBBd\textquotesingle,
\textquotesingle VBA\textquotesingle,
\textquotesingle KAE\textquotesingle,
\textquotesingle VAE\textquotesingle,
\textquotesingle KAA\textquotesingle,
\textquotesingle VAV\textquotesingle,
\textquotesingle KAV\textquotesingle{]},

{[}\textquotesingle KAV\textquotesingle,
\textquotesingle KBBd\textquotesingle,
\textquotesingle VBBd\textquotesingle,
\textquotesingle KBBd\textquotesingle,
\textquotesingle VAA\textquotesingle,
\textquotesingle KAV\textquotesingle{]},

{[}\textquotesingle KBG\textquotesingle,
\textquotesingle VBG\textquotesingle,
\textquotesingle KBBd\textquotesingle,
\textquotesingle VBBd\textquotesingle,
\textquotesingle KAA\textquotesingle{]},

{[}\textquotesingle KBBd\textquotesingle,
\textquotesingle VBBd\textquotesingle,
\textquotesingle KBA\textquotesingle,
\textquotesingle VAA\textquotesingle,
\textquotesingle KAA\textquotesingle{]},

{[}\textquotesingle KBG\textquotesingle,
\textquotesingle VBBd\textquotesingle,
\textquotesingle KBBd\textquotesingle,
\textquotesingle VBA\textquotesingle,
\textquotesingle VAA\textquotesingle,
\textquotesingle KAA\textquotesingle,
\textquotesingle VAV\textquotesingle,
\textquotesingle KAV\textquotesingle{]}

{]}

\emph{\# Übergangszählung initialisieren}

transitions \textbf{=} \{\}

\textbf{for} chain \textbf{in} empirical\_chains:

\textbf{for} i \textbf{in} range(len(chain) \textbf{-} 1):

start, end \textbf{=} chain{[}i{]}, chain{[}i \textbf{+} 1{]}

\textbf{if} start \textbf{not} \textbf{in} transitions:

transitions{[}start{]} \textbf{=} \{\}

\textbf{if} end \textbf{not} \textbf{in} transitions{[}start{]}:

transitions{[}start{]}{[}end{]} \textbf{=} 0

transitions{[}start{]}{[}end{]} \textbf{+=} 1

\emph{\# Normalisierung: Übergangswahrscheinlichkeiten berechnen}

probabilities \textbf{=} \{\}

\textbf{for} start \textbf{in} transitions:

total \textbf{=} sum(transitions{[}start{]}\textbf{.}values())

probabilities{[}start{]} \textbf{=} \{end: count \textbf{/} total
\textbf{for} end, count \textbf{in}
transitions{[}start{]}\textbf{.}items()\}

\emph{\# Terminalzeichen und Startzeichen definieren}

terminal\_symbols \textbf{=} list(set({[}item \textbf{for} sublist
\textbf{in} empirical\_chains \textbf{for} item \textbf{in} sublist{]}))

start\_symbol \textbf{=} empirical\_chains{[}0{]}{[}0{]}

\emph{\# Funktion zur Generierung von Ketten basierend auf der
Grammatik}

\textbf{def} generate\_chain(max\_length\textbf{=}10):

chain \textbf{=} {[}start\_symbol{]}

\textbf{while} len(chain) \textbf{\textless{}} max\_length:

current \textbf{=} chain{[}\textbf{-}1{]}

\textbf{if} current \textbf{not} \textbf{in} probabilities:

\textbf{break}

next\_symbol \textbf{=}
np\textbf{.}random\textbf{.}choice(list(probabilities{[}current{]}\textbf{.}keys()),
p\textbf{=}list(probabilities{[}current{]}\textbf{.}values()))

chain\textbf{.}append(next\_symbol)

\textbf{if} next\_symbol \textbf{not} \textbf{in} probabilities:

\textbf{break}

\textbf{return} chain

\emph{\# Funktion zur Berechnung relativer Häufigkeiten}

\textbf{def} compute\_frequencies(chains, terminals):

frequency\_array \textbf{=} np\textbf{.}zeros(len(terminals))

terminal\_index \textbf{=} \{term: i \textbf{for} i, term \textbf{in}
enumerate(terminals)\}

\textbf{for} chain \textbf{in} chains:

\textbf{for} symbol \textbf{in} chain:

\textbf{if} symbol \textbf{in} terminal\_index:

frequency\_array{[}terminal\_index{[}symbol{]}{]} \textbf{+=} 1

total \textbf{=} frequency\_array\textbf{.}sum()

\textbf{if} total \textbf{\textgreater{}} 0:

frequency\_array \textbf{/=} total \emph{\# Normierung der Häufigkeiten}

\textbf{return} frequency\_array

\emph{\# Iterative Optimierung}

max\_iterations \textbf{=} 1000

tolerance \textbf{=} 0.01 \emph{\# Toleranz für Standardmessfehler}

best\_correlation \textbf{=} 0

best\_significance \textbf{=} 1

\emph{\# Relativ Häufigkeiten der empirischen Ketten berechnen}

empirical\_frequencies \textbf{=}
compute\_frequencies(empirical\_chains, terminal\_symbols)

\textbf{for} iteration \textbf{in} range(max\_iterations):

\emph{\# Generiere 8 künstliche Ketten}

generated\_chains \textbf{=} {[}generate\_chain() \textbf{for} \_
\textbf{in} range(8){]}

\emph{\# Relativ Häufigkeiten der generierten Ketten berechnen}

generated\_frequencies \textbf{=}
compute\_frequencies(generated\_chains, terminal\_symbols)

\emph{\# Berechne die Korrelation}

correlation, p\_value \textbf{=} pearsonr(empirical\_frequencies,
generated\_frequencies)

print(f"Iteration \{iteration \textbf{+} 1\}, Korrelation:
\{correlation:.3f\}, Signifikanz: \{p\_value:.3f\}")

\emph{\# Überprüfen, ob Korrelation und Signifikanz akzeptabel sind}

\textbf{if} correlation \textbf{\textgreater=} 0.9 \textbf{and} p\_value
\textbf{\textless{}} 0.05:

best\_correlation \textbf{=} correlation

best\_significance \textbf{=} p\_value

\textbf{break}

\emph{\# Anpassung der Wahrscheinlichkeiten basierend auf
Standardmessfehler}

\textbf{for} start \textbf{in} probabilities:

\textbf{for} end \textbf{in} probabilities{[}start{]}:

\emph{\# Fehlerabschätzung basierend auf Differenz der Häufigkeiten}

empirical\_prob \textbf{=}
empirical\_frequencies{[}terminal\_symbols\textbf{.}index(end){]}

generated\_prob \textbf{=}
generated\_frequencies{[}terminal\_symbols\textbf{.}index(end){]}

error \textbf{=} empirical\_prob \textbf{-} generated\_prob

\emph{\# Anpassung der Wahrscheinlichkeit}

probabilities{[}start{]}{[}end{]} \textbf{+=} error \textbf{*} tolerance

probabilities{[}start{]}{[}end{]} \textbf{=} max(0, min(1,
probabilities{[}start{]}{[}end{]})) \emph{\# Begrenzen auf {[}0,1{]}}

\emph{\# Normalisierung}

\textbf{for} start \textbf{in} probabilities:

total \textbf{=} sum(probabilities{[}start{]}\textbf{.}values())

\textbf{if} total \textbf{\textgreater{}} 0:

probabilities{[}start{]} \textbf{=} \{end: prob \textbf{/} total
\textbf{for} end, prob \textbf{in}
probabilities{[}start{]}\textbf{.}items()\}

\emph{\# Ergebnis ausgeben}

print("\textbackslash nOptimierte probabilistische Grammatik:")

\textbf{for} start, transitions \textbf{in}
probabilities\textbf{.}items():

print(f"\{start\} → \{transitions\}")

print(f"\textbackslash nBeste Korrelation: \{best\_correlation:.3f\},
Signifikanz: \{best\_significance:.3f\}")

Iteration 1, Korrelation: 0.925, Signifikanz: 0.000

Optimierte probabilistische Grammatik:

KBG → \{\textquotesingle VBG\textquotesingle: 0.6666666666666666,
\textquotesingle VBBd\textquotesingle: 0.3333333333333333\}

VBG → \{\textquotesingle KBBd\textquotesingle: 1.0\}

KBBd → \{\textquotesingle VBBd\textquotesingle: 0.6666666666666666,
\textquotesingle VAA\textquotesingle: 0.16666666666666666,
\textquotesingle VBA\textquotesingle: 0.16666666666666666\}

VBBd → \{\textquotesingle KBA\textquotesingle: 0.4444444444444444,
\textquotesingle VAA\textquotesingle: 0.2222222222222222,
\textquotesingle KBBd\textquotesingle: 0.2222222222222222,
\textquotesingle KAA\textquotesingle: 0.1111111111111111\}

KBA → \{\textquotesingle VBA\textquotesingle: 0.5,
\textquotesingle VAA\textquotesingle: 0.5\}

VBA → \{\textquotesingle KBBd\textquotesingle: 0.5,
\textquotesingle KAE\textquotesingle: 0.25,
\textquotesingle VAA\textquotesingle: 0.25\}

VAA → \{\textquotesingle KAA\textquotesingle: 0.8571428571428571,
\textquotesingle KAV\textquotesingle: 0.14285714285714285\}

KAA → \{\textquotesingle VAV\textquotesingle: 0.75,
\textquotesingle VBG\textquotesingle: 0.25\}

VAV → \{\textquotesingle KAV\textquotesingle: 1.0\}

KAE → \{\textquotesingle VAE\textquotesingle: 1.0\}

VAE → \{\textquotesingle KAA\textquotesingle: 1.0\}

KAV → \{\textquotesingle KBBd\textquotesingle: 1.0\}

Beste Korrelation: 0.925, Signifikanz: 0.000

\end{document}
