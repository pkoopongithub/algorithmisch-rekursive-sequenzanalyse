% Options for packages loaded elsewhere
\PassOptionsToPackage{unicode}{hyperref}
\PassOptionsToPackage{hyphens}{url}
\documentclass[
]{article}
\usepackage{xcolor}
\usepackage{amsmath,amssymb}
\setcounter{secnumdepth}{-\maxdimen} % remove section numbering
\usepackage{iftex}
\ifPDFTeX
  \usepackage[T1]{fontenc}
  \usepackage[utf8]{inputenc}
  \usepackage{textcomp} % provide euro and other symbols
\else % if luatex or xetex
  \usepackage{unicode-math} % this also loads fontspec
  \defaultfontfeatures{Scale=MatchLowercase}
  \defaultfontfeatures[\rmfamily]{Ligatures=TeX,Scale=1}
\fi
\usepackage{lmodern}
\ifPDFTeX\else
  % xetex/luatex font selection
\fi
% Use upquote if available, for straight quotes in verbatim environments
\IfFileExists{upquote.sty}{\usepackage{upquote}}{}
\IfFileExists{microtype.sty}{% use microtype if available
  \usepackage[]{microtype}
  \UseMicrotypeSet[protrusion]{basicmath} % disable protrusion for tt fonts
}{}
\makeatletter
\@ifundefined{KOMAClassName}{% if non-KOMA class
  \IfFileExists{parskip.sty}{%
    \usepackage{parskip}
  }{% else
    \setlength{\parindent}{0pt}
    \setlength{\parskip}{6pt plus 2pt minus 1pt}}
}{% if KOMA class
  \KOMAoptions{parskip=half}}
\makeatother
\setlength{\emergencystretch}{3em} % prevent overfull lines
\providecommand{\tightlist}{%
  \setlength{\itemsep}{0pt}\setlength{\parskip}{0pt}}
\usepackage{bookmark}
\IfFileExists{xurl.sty}{\usepackage{xurl}}{} % add URL line breaks if available
\urlstyle{same}
\hypersetup{
  hidelinks,
  pdfcreator={LaTeX via pandoc}}

\author{}
\date{}

\begin{document}

\section{Algorithmic Recursive Sequence Analysis
2.0}\label{algorithmic-recursive-sequence-analysis-2.0}

\subsection{Grammar induction from further transcripts after extensive
sequence analysis of a
transcript}\label{grammar-induction-from-further-transcripts-after-extensive-sequence-analysis-of-a-transcript}

This document presents the method of algorithmically recursive sequence
analysis. The goal of this method is to extract grammatical structures
from natural language sequences to extract and then induce a grammar
based on it. We focus on them Analysis of transcripts prepared through
extensive sequence analysis to recognize the underlying rules and
patterns in the data.

\subsubsection{Background}\label{background}

Sequence analysis in natural language data is a widely used approach in
the Language processing to identify recurring patterns, structures and
dependencies between to identify different parts of a sequence. In this
extended version (2.0) we integrate grammatical induction as a way from
the analyzed to derive a formal grammar from sequences.

\subsubsection{Goal setting}\label{goal-setting}

The main objectives of this approach include:

\begin{enumerate}
\def\labelenumi{\arabic{enumi}.}
\item
  \textbf{Extension of sequence analysis}: The analysis is applied to a
  broader set of transcripts to test whether the discovered patterns can
  be generalized beyond a single transcript.
\item
  \textbf{Grammar induction}: The knowledge gained from the sequences is
  used to extract grammatical rules and develop formalized grammars.
\item
  \textbf{Quality Evaluation}: The quality of the induced grammar is
  evaluated based on its ability to generate future sequences that match
  the patterns in the transcripts.
\end{enumerate}

\subsubsection{Methodology}\label{methodology}

The method consists of several steps that are carried out iteratively:

\begin{enumerate}
\def\labelenumi{\arabic{enumi}.}
\item
  \textbf{Data preparation}: Collection and preprocessing of
  transcripts.
\item
  \textbf{Sequence analysis}: Carrying out an extensive sequence
  analysis with reading production and reading falsification on a
  transcript (loc. cit. in this repository). Conduct an extensive
  analysis of the transcripts, identifying statistical and syntactic
  patterns.
\item
  \textbf{Grammar induction}: Based on the identified patterns, a formal
  grammar (Python here in the document) is induced that can describe the
  generated sequences.
\item
  \textbf{Optimization}: The grammar is optimized (Python here in the
  document) and validated to increase its accuracy and generalizability.
\end{enumerate}

After evaluating specialist literature on sales discussions, the
following preliminary hypothetical grammar emerges: A sales conversation
(VKG) consists of a greeting (BG), a sales part (VT) and a farewell
(AV).

\begin{itemize}
\item
  The greeting consists of a greeting from the customer (KBG) and a
  greeting from the seller (VBG).
\item
  The farewell consists of a farewell by the customer (KAV) and a
  farewell by the seller (VAV).
\item
  The sales part consists of a requirements part (B) and a final part
  (A).

  \begin{itemize}
  \item
    The needs part includes a needs clarification (BBd) and a needs
    argument (BA).

    \begin{itemize}
    \item
      The needs clarification consists of the customer\textquotesingle s
      needs (KBBd) and the associated clarifications from the seller
      (VBBd).
    \item
      The needs argument consists of arguments from the customer (KBA)
      and the seller (VBA).
    \end{itemize}
  \item
    The closing part (A) consists of objections (AE) and a sales deal
    (AA).

    \begin{itemize}
    \item
      The objections consist of arguments from the customer (KAE) and
      the seller (VAE).
    \item
      The sales conclusion consists of arguments from the customer (KAA)
      and the seller (VAA).
    \end{itemize}
  \end{itemize}
\end{itemize}

The farewell (AV) consists of a farewell to the customer (KAV) and the
seller (VAV).

The start symbol is therefore VKG, and the terminal characters for the
categories are: KBG, VBG, KBBd, VBBd, KBA, VBA, KAE, VAE, KAA, VAA, KAV
and VAV. From this information, create a probabilistic context-free
grammar with initially assumed transition probabilities.

First, a transcript was created from an intensive sequence analysis with
readings production and Reading falsification, category formation,
intercoding correlation and grammar induction with schema and grammar
transduction with Lisp and parsing the Terminal chain carried out with
Object Pascal (loc. cit. in this repository).

The procedure can be divided into several clearly defined steps.
Here\textquotesingle s an overview to make sure all steps are
understandable:

\subsubsection{\texorpdfstring{1. \textbf{Formation of hypotheses about
grammar}}{1. Formation of hypotheses about grammar}}\label{formation-of-hypotheses-about-grammar}

\begin{itemize}
\item
  Conduct a preliminary investigation to create a \textbf{hypothetical
  PCFG}. This serves as a guide, but is not binding for the analysis of
  the transcripts.
\end{itemize}

\subsubsection{\texorpdfstring{2. \textbf{Analysis of the
transcripts}}{2. Analysis of the transcripts}}\label{analysis-of-the-transcripts}

\begin{itemize}
\item
  Each \textbf{transcript is analyzed individually}. The intentions of
  the speakers and the structure of the interactions are identified.
\item
  Each interaction in the transcript is assigned a \textbf{terminal
  character}. This results in a \textbf{terminal string} for each
  transcript.
\end{itemize}

\subsubsection{\texorpdfstring{3. \textbf{Induction of a grammar for
each
transcript}}{3. Induction of a grammar for each transcript}}\label{induction-of-a-grammar-for-each-transcript}

\begin{itemize}
\item
  A specific \textbf{grammar is induced} from the terminal string of
  each transcript.
\item
  There is a separate \textbf{grammar per transcript} at the end (eight
  in total).
\end{itemize}

\subsubsection{\texorpdfstring{4. \textbf{Unification of
Grammars}}{4. Unification of Grammars}}\label{unification-of-grammars}

\begin{itemize}
\item
  The eight individual grammars are merged into a \textbf{unified
  grammar} that covers the structure of all transcripts.
\end{itemize}

\subsubsection{\texorpdfstring{5. \textbf{Parsing terminal
strings}}{5. Parsing terminal strings}}\label{parsing-terminal-strings}

\begin{itemize}
\item
  The eight terminal strings of the transcripts are checked for
  \textbf{well-formedness} with respect to the unified grammar. This
  means that each string should be correctly recognized and parsed by
  the grammar.
\end{itemize}

\subsubsection{\texorpdfstring{6. \textbf{Optimization of
PCFG}}{6. Optimization of PCFG}}\label{optimization-of-pcfg}

\begin{itemize}
\item
  A first version of a PCFG is created from the previous analyses.
\item
  With the created PCFG, \textbf{artificial terminal strings} are
  created.
\item
  The \textbf{proportional frequency of the terminal characters} from
  the artificial strings is compared with the frequency of the terminal
  characters in the real data (the eight transcripts).
\item
  The \textbf{significance} of this correlation is measured.
\item
  Adjustments to the PCFG are made until the correlation is
  satisfactory.
\end{itemize}

\subsubsection{\texorpdfstring{7. \textbf{Repeat
Process}}{7. Repeat Process}}\label{repeat-process}

\begin{itemize}
\item
  Steps 6 and 7 are repeated until there is a good fit between the
  \textbf{empirical data} and the generated PCFG.
\end{itemize}

Here is a probabilistic context-free grammar (PCFG) based on structure.
In a PCFG, probabilities for transitions between different production
rules are defined. Since you just provided a structure without specific
probabilities, I\textquotesingle ll assume equal probabilities for the
possible options within the same level to provide a baseline. These
probabilities can be adjusted later when empirical data becomes
available.

\section{Probabilistische hypothetische kontextfreie Grammatik (PCFG)
für ein
Verkaufsgespräch}\label{probabilistische-hypothetische-kontextfreie-grammatik-pcfg-fuxfcr-ein-verkaufsgespruxe4ch}

VKG -\textgreater{} BG VT AV {[}1.0{]}

\section{Begrüßung}\label{begruxfcuxdfung}

BG -\textgreater{} KBG VBG {[}1.0{]} KBG -\textgreater{}
\textquotesingle Kunden-Gruß\textquotesingle{} VBG -\textgreater{}
\textquotesingle Verkäufer-Gruß\textquotesingle{}

\section{Verabschiedung}\label{verabschiedung}

AV -\textgreater{} KAV VAV {[}1.0{]} KAV -\textgreater{}
\textquotesingle Kunden-Verabschiedung\textquotesingle{} VAV
-\textgreater{}
\textquotesingle Verkäufer-Verabschiedung\textquotesingle{}

\section{Verkaufsteil}\label{verkaufsteil}

VT -\textgreater{} B A {[}1.0{]}

\section{Bedarfsteil}\label{bedarfsteil}

B -\textgreater{} BBd BA {[}1.0{]}

\section{Bedarfsklärung}\label{bedarfskluxe4rung}

BBd -\textgreater{} KBBd VBBd {[}1.0{]} KBBd -\textgreater{}
\textquotesingle Kunden-Bedarf\textquotesingle{} VBBd -\textgreater{}
\textquotesingle Verkäufer-Klärung\textquotesingle{}

\section{Bedarfsargumentation}\label{bedarfsargumentation}

BA -\textgreater{} KBA VBA {[}0.5{]} \textbar{} VBA KBA {[}0.5{]} KBA
-\textgreater{} \textquotesingle Kunden-Argument\textquotesingle{} VBA
-\textgreater{} \textquotesingle Verkäufer-Argument\textquotesingle{}

\section{Abschlussteil}\label{abschlussteil}

A -\textgreater{} AE AA {[}1.0{]}

\section{Einwände}\label{einwuxe4nde}

AE -\textgreater{} KAE VAE {[}0.5{]} \textbar{} VAE KAE {[}0.5{]} KAE
-\textgreater{} \textquotesingle Kunden-Einwand\textquotesingle{} VAE
-\textgreater{} \textquotesingle Verkäufer-Einwand\textquotesingle{}

\section{Verkaufsabschluss}\label{verkaufsabschluss}

AA -\textgreater{} KAA VAA {[}0.5{]} \textbar{} VAA KAA {[}0.5{]} KAA
-\textgreater{} \textquotesingle Kunden-Abschluss\textquotesingle{} VAA
-\textgreater{} \textquotesingle Verkäufer-Abschluss\textquotesingle{}

The eight Transcripts (Audio file here in Repository).

\subsubsection{\texorpdfstring{\textbf{Text 1}}{Text 1}}\label{text-1}

\textbf{Datum:} 28. Juni 1994, \textbf{Ort:} Metzgerei, Aachen, 11:00
Uhr

\emph{(Die Geräusche eines geschäftigen Marktplatzes im Hintergrund,
Stimmen und Gemurmel)}

\textbf{Verkäuferin:} Guten Tag, was darf es sein?

\textbf{Kunde:} Einmal von der groben Leberwurst, bitte.

\textbf{Verkäuferin:} Wie viel darf's denn sein?

\textbf{Kunde:} Zwei hundert Gramm.

\textbf{Verkäuferin:} Zwei hundert Gramm. Sonst noch etwas?

\textbf{Kunde:} Ja, dann noch ein Stück von dem Schwarzwälder Schinken.

\textbf{Verkäuferin:} Wie groß soll das Stück sein?

\textbf{Kunde:} So um die dreihundert Gramm.

\textbf{Verkäuferin:} Alles klar. Kommt sofort. \emph{(Geräusche von
Papier und Verpackung)}

\textbf{Kunde:} Danke schön.

\textbf{Verkäuferin:} Das macht dann acht Mark zwanzig.

\textbf{Kunde:} Bitte. \emph{(Klimpern von Münzen, Geräusche der Kasse)}

\textbf{Verkäuferin:} Danke und einen schönen Tag noch!

\textbf{Kunde:} Danke, ebenfalls!

\textbf{Ende Text 1}

\subsubsection{\texorpdfstring{\textbf{Text 2}}{Text 2}}\label{text-2}

\textbf{Datum:} 28. Juni 1994, \textbf{Ort:} Marktplatz, Aachen

\emph{(Ständige Hintergrundgeräusche von Stimmen und Marktatmosphäre)}

\textbf{Verkäufer:} Kirschen kann jeder probieren hier, Kirschen kann
jeder probieren hier!

\textbf{Kunde 1:} Ein halbes Kilo Kirschen, bitte.

\textbf{Verkäufer:} Ein halbes Kilo? Oder ein Kilo?

\emph{(Unverständliches Gespräch, Münzen klimpern)}

\textbf{Verkäufer:} Danke schön!

\textbf{Verkäufer:} Kirschen kann jeder probieren hier! Drei Mark,
bitte.

\textbf{Kunde 1:} Danke schön!

\textbf{Verkäufer:} Kirschen kann jeder probieren hier, Kirschen kann
jeder probieren hier!

\emph{(Weitere Stimmen im Hintergrund, unverständliches Gespräch, Münzen
klimpern)}

\textbf{Kunde 2:} Ein halbes Kilo, bitte.

\emph{(Unverständliches Gespräch)}

\textbf{Ende Text 2}

\subsubsection{\texorpdfstring{\textbf{Text 3}}{Text 3}}\label{text-3}

\textbf{Datum:} 28. Juni 1994, \textbf{Ort:} Fischstand, Marktplatz,
Aachen

\emph{(Marktatmosphäre, Gespräch im Hintergrund, teilweise
unverständlich)}

\textbf{Kunde:} Ein Pfund Seelachs, bitte.

\textbf{Verkäufer:} Seelachs, alles klar.

\emph{(Geräusche von Verpackung und Verkaufsvorbereitungen)}

\textbf{Verkäufer:} Vier Mark neunzehn, bitte.

\emph{(Geräusche von Verpackung, Münzen klimpern)}

\textbf{Verkäufer:} Schönen Dank!

\textbf{Kunde:} Ja, danke schön!

\textbf{Ende Text 3}

\subsubsection{\texorpdfstring{\textbf{Text 4}}{Text 4}}\label{text-4}

\textbf{Datum:} 28. Juni 1994, \textbf{Ort:} Gemüsestand, Aachen,
Marktplatz, 11:00 Uhr

\emph{(Marktatmosphäre, teilweise unverständlich)}

\textbf{Kunde:} Hören Sie, ich nehme ein paar Champignons mit.

\textbf{Verkäufer:} Braune oder helle?

\textbf{Kunde:} Nehmen wir die hellen.

\textbf{Verkäufer:} Alles klar, die hellen.

\emph{(Unverständliche Unterhaltung im Hintergrund)}

\textbf{Verkäufer:} Die sind beide frisch, keine Sorge.

\textbf{Kunde:} Wie ist es mit Pfifferlingen?

\textbf{Verkäufer:} Ah, die sind super!

\emph{(Unverständliches Gespräch)}

\textbf{Kunde:} Kann ich die in Reissalat tun?

\textbf{Verkäufer:} Eher kurz anbraten in der Pfanne.

\textbf{Kunde:} Okay, mache ich.

\textbf{Verkäufer:} Die können Sie roh verwenden, aber ein bisschen
anbraten ist besser.

\textbf{Kunde:} Verstanden.

\emph{(Weitere Unterhaltung, unverständliche Kommentare)}

\textbf{Verkäufer:} Noch etwas anderes?

\textbf{Kunde:} Ja, dann nehme ich noch Erdbeeren.

\emph{(Pause, Hintergrundgeräusche von Verpackung und Stimmen)}

\textbf{Verkäufer:} Schönen Tag noch!

\textbf{Kunde:} Gleichfalls!

\textbf{Ende Text 4}

\subsubsection{\texorpdfstring{\textbf{Text 5}}{Text 5}}\label{text-5}

\textbf{Datum:} 26. Juni 1994, \textbf{Ort:} Gemüsestand, Aachen,
Marktplatz, 11:00 Uhr

\emph{(Marktatmosphäre, teilweise unverständlich)}

\textbf{Verkäufer:} So, bitte schön.

\textbf{Kunde 1:} Auf Wiedersehen!

\textbf{Kunde 2:} Ich hätte gern ein Kilo von den Granny Smith Äpfeln
hier.

\emph{(Unverständliches Gespräch im Hintergrund)}

\textbf{Verkäufer:} Sonst noch etwas?

\textbf{Kunde 2:} Ja, noch ein Kilo Zwiebeln.

\textbf{Verkäufer:} Alles klar.

\emph{(Unverständliches Gespräch, Hintergrundgeräusche)}

\textbf{Kunde 2:} Das war\textquotesingle s.

\textbf{Verkäufer:} Sechs Mark fünfundzwanzig, bitte.

\emph{(Unverständliches Gespräch, Geräusche von Münzen und Verpackung)}

\textbf{Verkäufer:} Wiedersehen!

\textbf{Kunde 2:} Wiedersehen!

\textbf{Ende Text 5}

\subsubsection{\texorpdfstring{\textbf{Text 6}}{Text 6}}\label{text-6}

\textbf{Datum:} 28. Juni 1994, \textbf{Ort:} Käseverkaufsstand, Aachen,
Marktplatz

\emph{(Marktatmosphäre, Begrüßungen)}

\textbf{Kunde 1:} Guten Morgen!

\textbf{Verkäufer:} Guten Morgen!

\textbf{Kunde 1:} Ich hätte gerne fünfhundert Gramm holländischen Gouda.

\textbf{Verkäufer:} Am Stück?

\textbf{Kunde 1:} Ja, am Stück, bitte.

\textbf{Ende Text 6}

\subsubsection{\texorpdfstring{\textbf{Text 7}}{Text 7}}\label{text-7}

\textbf{Datum:} 28. Juni 1994, \textbf{Ort:} Bonbonstand, Aachen,
Marktplatz, 11:30 Uhr

\emph{(Geräusche von Stimmen und Marktatmosphäre, teilweise
unverständlich)}

\textbf{Kunde:} Von den gemischten hätte ich gerne hundert Gramm.

\emph{(Unverständliche Fragen und Antworten)}

\textbf{Verkäufer:} Für zu Hause oder zum Mitnehmen?

\textbf{Kunde:} Zum Mitnehmen, bitte.

\textbf{Verkäufer:} Fünfzig Pfennig, bitte.

\emph{(Klimpern von Münzen, Geräusche von Verpackung)}

\textbf{Kunde:} Danke!

\textbf{Ende Text 7}

\subsubsection{\texorpdfstring{\textbf{Text 8}}{Text 8}}\label{text-8}

\textbf{Datum:} 9. Juli 1994, \textbf{Ort:} Bäckerei, Aachen, 12:00 Uhr

\emph{(Schritte hörbar, Hintergrundgeräusche, teilweise unverständlich)}

\textbf{Kunde:} Guten Tag!

\emph{(Unverständliche Begrüßung im Hintergrund)}

\textbf{Verkäuferin:} Einmal unser bester Kaffee, frisch gemahlen,
bitte.

\emph{(Geräusche der Kaffeemühle, Verpackungsgeräusche)}

\textbf{Verkäuferin:} Sonst noch etwas?

\textbf{Kunde:} Ja, noch zwei Stück Obstsalat und ein Schälchen Sahne.

\textbf{Verkäuferin:} In Ordnung!

\emph{(Geräusche der Kaffeemühle, Papiergeräusche)}

\textbf{Verkäuferin:} Ein kleines Schälchen Sahne, ja?

\textbf{Kunde:} Ja, danke.

\emph{(Türgeräusch, Lachen, Papiergeräusche)}

\textbf{Verkäuferin:} Keiner kümmert sich darum, die Türen zu ölen.

\textbf{Kunde:} Ja, das ist immer so.

\emph{(Lachen, Geräusche von Münzen und Verpackung)}

\textbf{Verkäuferin:} Das macht vierzehn Mark und neunzehn Pfennig,
bitte.

\textbf{Kunde:} Ich zahle in Kleingeld.

\emph{(Lachen und Geräusche von Münzen)}

\textbf{Verkäuferin:} Vielen Dank, schönen Sonntag noch!

\textbf{Kunde:} Danke, Ihnen auch!

\textbf{Ende Text 8}

Auf Grundlage der vorläufigen probabilistischen kontextfreien Grammatik
(PCFG) werden die Transkripte in die entsprechenden Terminalzeichen
umwandeln. Hier ist die detaillierte Zuordnung für jedes Transkript,
wobei für jede relevante Aktion oder Aussage des Gesprächs ein passendes
Terminalzeichen aus der vorgegebenen Grammatik verwende.

\subsubsection{\texorpdfstring{\textbf{Transkript 1 -
Terminalzeichen}}{Transkript 1 - Terminalzeichen}}\label{transkript-1---terminalzeichen}

\textbf{Datum:} 28. Juni 1994, \textbf{Ort:} Metzgerei, Aachen, 11:00
Uhr\textbar{} Kunde: Guten Tag \textbar{} KBG (Kunden-Gruß) \textbar{}
Verkäuferin: Guten Tag \textbar{} VBG (Verkäufer-Gruß) \textbar{} Kunde:
Einmal von der groben Leberwurst, bitte. \textbar{} KBBd (Kunden-Bedarf)
\textbar{} Verkäuferin: Wie viel darf's denn sein? \textbar{} VBBd
(Verkäufer-Klärung) \textbar{} Kunde: Zwei hundert Gramm. \textbar{} KBA
(Kunden-Argument) \textbar{} Verkäuferin: Sonst noch etwas? \textbar{}
VBA (Verkäufer-Argument) \textbar{} Kunde: Ja, dann noch ein Stück von
dem Schwarzwälde\textbar{} KBBd (Kunden-Bedarf) \textbar{} Verkäuferin:
Wie groß soll das Stück sein? \textbar{} VBBd (Verkäufer-Klärung)
\textbar{} Kunde: So um die dreihundert Gramm. \textbar{} KBA
(Kunden-Argument) \textbar{} Verkäuferin: Das macht dann acht Mark
zwanzig. \textbar{} VAA (Verkäufer-Abschluss) \textbar{} Kunde: Bitte.
\textbar{} KAA (Kunden-Abschluss) \textbar{} Verkäuferin: Danke und
einen schönen Tag noch! \textbar{} VAV (Verkäufer-Verabschiedung)
\textbar{} Kunde: Danke, ebenfalls! \textbar{} KAV
(Kunden-Verabschiedung)\textbar{}

\subsubsection{\texorpdfstring{\textbf{Transkript 2 -
Terminalzeichen}}{Transkript 2 - Terminalzeichen}}\label{transkript-2---terminalzeichen}

\textbf{Datum:} 28. Juni 1994, \textbf{Ort:} Marktplatz,
Aachen\textbar{} Verkäufer: Kirschen kann jeder probieren hier!
\textbar{} VBG (Verkäufer-Gruß) \textbar{} Kunde 1: Ein halbes Kilo
Kirschen, bitte. \textbar{} KBBd (Kunden-Bedarf) \textbar{} Verkäufer:
Ein halbes Kilo? Oder ein Kilo? \textbar{} VBBd (Verkäufer-Klärung)
\textbar{} Verkäufer: Drei Mark, bitte. \textbar{} VAA
(Verkäufer-Abschluss) \textbar{} Kunde 1: Danke schön! \textbar{} KAA
(Kunden-Abschluss) \textbar{} Verkäufer: Kirschen kann jeder probieren
hier! \textbar{} VBG (Verkäufer-Gruß) \textbar{} Kunde 2: Ein halbes
Kilo, bitte. \textbar{} KBBd (Kunden-Bedarf) \textbar{} Verkäufer: Drei
Mark, bitte. \textbar{} VAA (Verkäufer-Abschluss) \textbar{} Kunde 2:
Danke schön! \textbar{} KAA (Kunden-Abschluss)

\subsubsection{\texorpdfstring{\textbf{Transkript 3 -
Terminalzeichen}}{Transkript 3 - Terminalzeichen}}\label{transkript-3---terminalzeichen}

\textbf{Datum:} 28. Juni 1994, \textbf{Ort:} Fischstand, Marktplatz,
Aachen\textbar{} Kunde: Ein Pfund Seelachs, bitte. \textbar{} KBBd
(Kunden-Bedarf) \textbar{} Verkäufer: Seelachs, alles klar. \textbar{}
VBBd (Verkäufer-Klärung) \textbar{} Verkäufer: Vier Mark neunzehn,
bitte. \textbar{} VAA (Verkäufer-Abschluss) \textbar{} Kunde: Danke
schön! \textbar{} KAA (Kunden-Abschluss)

\subsubsection{\texorpdfstring{\textbf{Transkript 4 -
Terminalzeichen}}{Transkript 4 - Terminalzeichen}}\label{transkript-4---terminalzeichen}

\textbf{Datum:} 28. Juni 1994, \textbf{Ort:} Gemüsestand, Aachen,
Marktplatz, 11:00 Uhr\textbar{} Kunde: Hören Sie, ich nehme ein paar
Champignons mit. \textbar{} KBBd (Kunden-Bedarf) \textbar{} Verkäufer:
Braune oder helle? \textbar{} VBBd (Verkäufer-Klärung) \textbar{} Kunde:
Nehmen wir die hellen. \textbar{} KBA (Kunden-Argument) \textbar{}
Verkäufer: Die sind beide frisch, keine Sorge. \textbar{} VBA
(Verkäufer-Argument) \textbar{} Kunde: Wie ist es mit Pfifferlingen?
\textbar{} KBBd (Kunden-Bedarf) \textbar{} Verkäufer: Ah, die sind
super! \textbar{} VBA (Verkäufer-Argument) \textbar{} Kunde: Kann ich
die in Reissalat tun? \textbar{} KAE (Kunden-Einwand) \textbar{}
Verkäufer: Eher kurz anbraten in der Pfanne. \textbar{} VAE
(Verkäufer-Einwand) \textbar{} Kunde: Okay, mache ich. \textbar{} KAA
(Kunden-Abschluss) \textbar{} Verkäufer: Schönen Tag noch! \textbar{}
VAV (Verkäufer-Verabschiedung) \textbar{} Kunde: Gleichfalls! \textbar{}
KAV (Kunden-Verabschiedung)

\subsubsection{\texorpdfstring{\textbf{Transkript 5 -
Terminalzeichen}}{Transkript 5 - Terminalzeichen}}\label{transkript-5---terminalzeichen}

\textbf{Datum:} 26. Juni 1994, \textbf{Ort:} Gemüsestand, Aachen,
Marktplatz, 11:00 Uhr\textbar{} Kunde 1: Auf Wiedersehen! \textbar{} KAV
(Kunden-Verabschiedung) \textbar{} Kunde 2: Ich hätte gern ein Kilo von
den Granny Smi\textbar{} KBBd (Kunden-Bedarf) \textbar{} Verkäufer:
Sonst noch etwas? \textbar{} VBBd (Verkäufer-Klärung) \textbar{} Kunde
2: Ja, noch ein Kilo Zwiebeln. \textbar{} KBBd (Kunden-Bedarf)
\textbar{} Verkäufer: Sechs Mark fünfundzwanzig, bitte. \textbar{} VAA
(Verkäufer-Abschluss) \textbar{} Kunde 2: Auf Wiedersehen! \textbar{}
KAV (Kunden-Verabschiedung)

\subsubsection{\texorpdfstring{\textbf{Transkript 6 -
Terminalzeichen}}{Transkript 6 - Terminalzeichen}}\label{transkript-6---terminalzeichen}

\textbf{Datum:} 28. Juni 1994, \textbf{Ort:} Käseverkaufsstand, Aachen,
Marktplatz\textbar{} Kunde 1: Guten Morgen! \textbar{} KBG (Kunden-Gruß)
\textbar{} Verkäufer: Guten Morgen! \textbar{} VBG (Verkäufer-Gruß)
\textbar{} Kunde 1: Ich hätte gerne fünfhundert Gramm holländi\textbar{}
KBBd (Kunden-Bedarf) \textbar{} Verkäufer: Am Stück? \textbar{} VBBd
(Verkäufer-Klärung) \textbar{} Kunde 1: Ja, am Stück, bitte. \textbar{}
KAA (Kunden-Abschluss)

\subsubsection{\texorpdfstring{\textbf{Transkript 7 -
Terminalzeichen}}{Transkript 7 - Terminalzeichen}}\label{transkript-7---terminalzeichen}

\textbf{Datum:} 28. Juni 1994, \textbf{Ort:} Bonbonstand, Aachen,
Marktplatz, 11:30 Uhr\textbar{} Kunde: Von den gemischten hätte ich
gerne hundert G\textbar{} KBBd (Kunden-Bedarf) \textbar{} Verkäufer: Für
zu Hause oder zum Mitnehmen? \textbar{} VBBd (Verkäufer-Klärung)
\textbar{} Kunde: Zum Mitnehmen, bitte. \textbar{} KBA (Kunden-Argument)
\textbar{} Verkäufer: Fünfzig Pfennig, bitte. \textbar{} VAA
(Verkäufer-Abschluss) \textbar{} Kunde: Danke! \textbar{} KAA
(Kunden-Abschluss)

\subsubsection{\texorpdfstring{\textbf{Transkript 8 -
Terminalzeichen}}{Transkript 8 - Terminalzeichen}}\label{transkript-8---terminalzeichen}

\textbf{Datum:} 9. Juli 1994, \textbf{Ort:} Bäckerei, Aachen, 12:00
Uhr\textbar{} Kunde: Guten Tag! \textbar{} KBG (Kunden-Gruß) \textbar{}
Verkäuferin: Einmal unser bester Kaffee, frisch gem\textbar{} VBBd
(Verkäufer-Klärung) \textbar{} Kunde: Ja, noch zwei Stück Obstsalat und
ein Schälc\textbar{} KBBd (Kunden-Bedarf) \textbar{} Verkäuferin: In
Ordnung! \textbar{} VBA (Verkäufer-Argument) \textbar{} Verkäuferin: Das
macht vierzehn Mark und neunzehn P\textbar{} VAA (Verkäufer-Abschluss)
\textbar{} Kunde: Ich zahle in Kleingeld. \textbar{} KAA
(Kunden-Abschluss) \textbar{} Verkäuferin: Vielen Dank, schönen Sonntag
noch! \textbar{} VAV (Verkäufer-Verabschiedung) \textbar{} Kunde: Danke,
Ihnen auch! \textbar{} KAV (Kunden-Verabschiedung

\textbf{import} numpy \textbf{as} np

\textbf{from} scipy.stats \textbf{import} pearsonr

\emph{\# Neue empirische Terminalzeichenketten}

empirical\_chains \textbf{=} {[}

{[}\textquotesingle KBG\textquotesingle,
\textquotesingle VBG\textquotesingle,
\textquotesingle KBBd\textquotesingle,
\textquotesingle VBBd\textquotesingle,
\textquotesingle KBA\textquotesingle,
\textquotesingle VBA\textquotesingle,
\textquotesingle KBBd\textquotesingle,
\textquotesingle VBBd\textquotesingle,
\textquotesingle KBA\textquotesingle,
\textquotesingle VAA\textquotesingle,
\textquotesingle KAA\textquotesingle,
\textquotesingle VAV\textquotesingle,
\textquotesingle KAV\textquotesingle{]},

{[}\textquotesingle VBG\textquotesingle,
\textquotesingle KBBd\textquotesingle,
\textquotesingle VBBd\textquotesingle,
\textquotesingle VAA\textquotesingle,
\textquotesingle KAA\textquotesingle,
\textquotesingle VBG\textquotesingle,
\textquotesingle KBBd\textquotesingle,
\textquotesingle VAA\textquotesingle,
\textquotesingle KAA\textquotesingle{]},

{[}\textquotesingle KBBd\textquotesingle,
\textquotesingle VBBd\textquotesingle,
\textquotesingle VAA\textquotesingle,
\textquotesingle KAA\textquotesingle{]},

{[}\textquotesingle KBBd\textquotesingle,
\textquotesingle VBBd\textquotesingle,
\textquotesingle KBA\textquotesingle,
\textquotesingle VBA\textquotesingle,
\textquotesingle KBBd\textquotesingle,
\textquotesingle VBA\textquotesingle,
\textquotesingle KAE\textquotesingle,
\textquotesingle VAE\textquotesingle,
\textquotesingle KAA\textquotesingle,
\textquotesingle VAV\textquotesingle,
\textquotesingle KAV\textquotesingle{]},

{[}\textquotesingle KAV\textquotesingle,
\textquotesingle KBBd\textquotesingle,
\textquotesingle VBBd\textquotesingle,
\textquotesingle KBBd\textquotesingle,
\textquotesingle VAA\textquotesingle,
\textquotesingle KAV\textquotesingle{]},

{[}\textquotesingle KBG\textquotesingle,
\textquotesingle VBG\textquotesingle,
\textquotesingle KBBd\textquotesingle,
\textquotesingle VBBd\textquotesingle,
\textquotesingle KAA\textquotesingle{]},

{[}\textquotesingle KBBd\textquotesingle,
\textquotesingle VBBd\textquotesingle,
\textquotesingle KBA\textquotesingle,
\textquotesingle VAA\textquotesingle,
\textquotesingle KAA\textquotesingle{]},

{[}\textquotesingle KBG\textquotesingle,
\textquotesingle VBBd\textquotesingle,
\textquotesingle KBBd\textquotesingle,
\textquotesingle VBA\textquotesingle,
\textquotesingle VAA\textquotesingle,
\textquotesingle KAA\textquotesingle,
\textquotesingle VAV\textquotesingle,
\textquotesingle KAV\textquotesingle{]}

{]}

\emph{\# Übergangszählung initialisieren}

transitions \textbf{=} \{\}

\textbf{for} chain \textbf{in} empirical\_chains:

\textbf{for} i \textbf{in} range(len(chain) \textbf{-} 1):

start, end \textbf{=} chain{[}i{]}, chain{[}i \textbf{+} 1{]}

\textbf{if} start \textbf{not} \textbf{in} transitions:

transitions{[}start{]} \textbf{=} \{\}

\textbf{if} end \textbf{not} \textbf{in} transitions{[}start{]}:

transitions{[}start{]}{[}end{]} \textbf{=} 0

transitions{[}start{]}{[}end{]} \textbf{+=} 1

\emph{\# Normalisierung: Übergangswahrscheinlichkeiten berechnen}

probabilities \textbf{=} \{\}

\textbf{for} start \textbf{in} transitions:

total \textbf{=} sum(transitions{[}start{]}\textbf{.}values())

probabilities{[}start{]} \textbf{=} \{end: count \textbf{/} total
\textbf{for} end, count \textbf{in}
transitions{[}start{]}\textbf{.}items()\}

\emph{\# Terminalzeichen und Startzeichen definieren}

terminal\_symbols \textbf{=} list(set({[}item \textbf{for} sublist
\textbf{in} empirical\_chains \textbf{for} item \textbf{in} sublist{]}))

start\_symbol \textbf{=} empirical\_chains{[}0{]}{[}0{]}

\emph{\# Funktion zur Generierung von Ketten basierend auf der
Grammatik}

\textbf{def} generate\_chain(max\_length\textbf{=}10):

chain \textbf{=} {[}start\_symbol{]}

\textbf{while} len(chain) \textbf{\textless{}} max\_length:

current \textbf{=} chain{[}\textbf{-}1{]}

\textbf{if} current \textbf{not} \textbf{in} probabilities:

\textbf{break}

next\_symbol \textbf{=}
np\textbf{.}random\textbf{.}choice(list(probabilities{[}current{]}\textbf{.}keys()),
p\textbf{=}list(probabilities{[}current{]}\textbf{.}values()))

chain\textbf{.}append(next\_symbol)

\textbf{if} next\_symbol \textbf{not} \textbf{in} probabilities:

\textbf{break}

\textbf{return} chain

\emph{\# Funktion zur Berechnung relativer Häufigkeiten}

\textbf{def} compute\_frequencies(chains, terminals):

frequency\_array \textbf{=} np\textbf{.}zeros(len(terminals))

terminal\_index \textbf{=} \{term: i \textbf{for} i, term \textbf{in}
enumerate(terminals)\}

\textbf{for} chain \textbf{in} chains:

\textbf{for} symbol \textbf{in} chain:

\textbf{if} symbol \textbf{in} terminal\_index:

frequency\_array{[}terminal\_index{[}symbol{]}{]} \textbf{+=} 1

total \textbf{=} frequency\_array\textbf{.}sum()

\textbf{if} total \textbf{\textgreater{}} 0:

frequency\_array \textbf{/=} total \emph{\# Normierung der Häufigkeiten}

\textbf{return} frequency\_array

\emph{\# Iterative Optimierung}

max\_iterations \textbf{=} 1000

tolerance \textbf{=} 0.01 \emph{\# Toleranz für Standardmessfehler}

best\_correlation \textbf{=} 0

best\_significance \textbf{=} 1

\emph{\# Relativ Häufigkeiten der empirischen Ketten berechnen}

empirical\_frequencies \textbf{=}
compute\_frequencies(empirical\_chains, terminal\_symbols)

\textbf{for} iteration \textbf{in} range(max\_iterations):

\emph{\# Generiere 8 künstliche Ketten}

generated\_chains \textbf{=} {[}generate\_chain() \textbf{for} \_
\textbf{in} range(8){]}

\emph{\# Relativ Häufigkeiten der generierten Ketten berechnen}

generated\_frequencies \textbf{=}
compute\_frequencies(generated\_chains, terminal\_symbols)

\emph{\# Berechne die Korrelation}

correlation, p\_value \textbf{=} pearsonr(empirical\_frequencies,
generated\_frequencies)

print(f"Iteration \{iteration \textbf{+} 1\}, Korrelation:
\{correlation:.3f\}, Signifikanz: \{p\_value:.3f\}")

\emph{\# Überprüfen, ob Korrelation und Signifikanz akzeptabel sind}

\textbf{if} correlation \textbf{\textgreater=} 0.9 \textbf{and} p\_value
\textbf{\textless{}} 0.05:

best\_correlation \textbf{=} correlation

best\_significance \textbf{=} p\_value

\textbf{break}

\emph{\# Anpassung der Wahrscheinlichkeiten basierend auf
Standardmessfehler}

\textbf{for} start \textbf{in} probabilities:

\textbf{for} end \textbf{in} probabilities{[}start{]}:

\emph{\# Fehlerabschätzung basierend auf Differenz der Häufigkeiten}

empirical\_prob \textbf{=}
empirical\_frequencies{[}terminal\_symbols\textbf{.}index(end){]}

generated\_prob \textbf{=}
generated\_frequencies{[}terminal\_symbols\textbf{.}index(end){]}

error \textbf{=} empirical\_prob \textbf{-} generated\_prob

\emph{\# Anpassung der Wahrscheinlichkeit}

probabilities{[}start{]}{[}end{]} \textbf{+=} error \textbf{*} tolerance

probabilities{[}start{]}{[}end{]} \textbf{=} max(0, min(1,
probabilities{[}start{]}{[}end{]})) \emph{\# Begrenzen auf {[}0,1{]}}

\emph{\# Normalisierung}

\textbf{for} start \textbf{in} probabilities:

total \textbf{=} sum(probabilities{[}start{]}\textbf{.}values())

\textbf{if} total \textbf{\textgreater{}} 0:

probabilities{[}start{]} \textbf{=} \{end: prob \textbf{/} total
\textbf{for} end, prob \textbf{in}
probabilities{[}start{]}\textbf{.}items()\}

\emph{\# Ergebnis ausgeben}

print("\textbackslash nOptimierte probabilistische Grammatik:")

\textbf{for} start, transitions \textbf{in}
probabilities\textbf{.}items():

print(f"\{start\} → \{transitions\}")

print(f"\textbackslash nBeste Korrelation: \{best\_correlation:.3f\},
Signifikanz: \{best\_significance:.3f\}")

Iteration 1, Korrelation: 0.861, Signifikanz: 0.000

Iteration 2, Korrelation: 0.835, Signifikanz: 0.001

Iteration 3, Korrelation: 0.903, Signifikanz: 0.000

Optimierte probabilistische Grammatik:

KBG → \{\textquotesingle VBG\textquotesingle: 0.6665949194957433,
\textquotesingle VBBd\textquotesingle: 0.33340508050425677\}

VBG → \{\textquotesingle KBBd\textquotesingle: 1.0\}

KBBd → \{\textquotesingle VBBd\textquotesingle: 0.6661627789328691,
\textquotesingle VAA\textquotesingle: 0.16672317719273486,
\textquotesingle VBA\textquotesingle: 0.16711404387439593\}

VBBd → \{\textquotesingle KBA\textquotesingle: 0.4444746815960467,
\textquotesingle VAA\textquotesingle: 0.22218925521330254,
\textquotesingle KBBd\textquotesingle: 0.2214541950661155,
\textquotesingle KAA\textquotesingle: 0.11188186812453531\}

KBA → \{\textquotesingle VBA\textquotesingle: 0.5001954483997749,
\textquotesingle VAA\textquotesingle: 0.4998045516002252\}

VBA → \{\textquotesingle KBBd\textquotesingle: 0.4995306939857028,
\textquotesingle KAE\textquotesingle: 0.25025106688027127,
\textquotesingle VAA\textquotesingle: 0.2502182391340259\}

VAA → \{\textquotesingle KAA\textquotesingle: 0.8573622459421577,
\textquotesingle KAV\textquotesingle: 0.1426377540578423\}

KAA → \{\textquotesingle VAV\textquotesingle: 0.750137275176534,
\textquotesingle VBG\textquotesingle: 0.24986272482346608\}

VAV → \{\textquotesingle KAV\textquotesingle: 1.0\}

KAE → \{\textquotesingle VAE\textquotesingle: 1.0\}

VAE → \{\textquotesingle KAA\textquotesingle: 1.0\}

KAV → \{\textquotesingle KBBd\textquotesingle: 1.0\}

Beste Korrelation: 0.903, Signifikanz: 0.000

\end{document}
